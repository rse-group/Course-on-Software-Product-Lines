% Keys to support piece-wise uncovering of elements in TikZ pictures:
% \node[visible on=<2->](foo){Foo}
% \node[visible on=<{2,4}>](bar){Bar}   % put braces around comma expressions
%
% Internally works by setting opacity=0 when invisible, which has the 
% adavantage (compared to \node<2->(foo){Foo} that the node is always there, hence
% always consumes space plus that coordinate (foo) is always available.
%
% The actual command that implements the invisibility can be overriden
% by altering the style invisible. For instance \tikzsset{invisible/.style={opacity=0.2}}
% would dim the "invisible" parts. Alternatively, the color might be set to white, if the
% output driver does not support transparencies (e.g., PS) 
%
\tikzset{
	invisible/.style={opacity=0},
	visible on/.style={alt={#1{}{invisible}}},
	alt/.code args={<#1>#2#3}{%
		\alt<#1>{\pgfkeysalso{#2}}{\pgfkeysalso{#3}} % \pgfkeysalso doesn't change the path
	},
}
