\documentclass[aspectratio=1610,10pt,handout]{beamer}
\setbeameroption{show notes}

%% Additional macro to read local beamer themes from a specific directory
\makeatletter
\def\beamer@calltheme#1#2#3{%
	\def\beamer@themelist{#2}
	\@for\beamer@themename:=\beamer@themelist\do
	{\usepackage[{#1}]{\beamer@themelocation/#3\beamer@themename}}}

\def\usefolder#1{
	\def\beamer@themelocation{#1}
}
\def\beamer@themelocation{}

\usefolder{Theme}
\usetheme[
%%% options passed to the outer theme
%    hidetitle,           % hide the (short) title in the sidebar
%    hideauthor,          % hide the (short) author in the sidebar
%    hideinstitute,       % hide the (short) institute in the bottom of the sidebar
%    shownavsym,          % show the navigation symbols
%    width=2cm,           % width of the sidebar (default is 2 cm)
%    hideothersubsections,% hide all subsections but the subsections in the current section
%    hideallsubsections,  % hide all subsections
    left               % right of left position of sidebar (default is right)
%%% options passed to the color theme
%    lightheaderbg,       % use a light header background
  ]{FasilkomUIsidebar}

% If you want to change the colors of the various elements in the theme, edit and uncomment the following lines
%%%% Template for Blue-Red Color theme %%%%
% Header color (Red)
\definecolor{beamer@headercolor}{RGB}{200,0,0}% red
%% Change the bar and sidebar colors:
\setbeamercolor{FasilkomUIsidebar}{fg=beamer@barcolor,bg=csui@red}
\setbeamercolor{sidebar}{bg=csui@blue}
%% Change the color of the structural elements:
\setbeamercolor{structure}{fg=gray}
%% Change the frame title text color:
\setbeamercolor{frametitle}{bg=csui@red}
%% Sidebar font color
\setbeamercolor{section in sidebar}{fg=white}
\setbeamercolor{section in sidebar shaded}{fg=gray}
\setbeamercolor{subsection in sidebar}{fg=white}
\setbeamercolor{subsection in sidebar shaded}{fg=gray}
%%%% End of - Template for Blue-Red Color theme %%%%
% ... and you can of course change a lot more - see the beamer user manual.


\usepackage[utf8]{inputenc}
\usepackage[english]{babel}
\usepackage[T1]{fontenc}
% Or whatever. Note that the encoding and the font should match. If T1
% does not look nice, try deleting the line with the fontenc.
\usepackage{helvet}

% colored hyperlinks
\newcommand{\chref}[2]{%
  \href{#1}{{\usebeamercolor[bg]{FasilkomUIsidebar}#2}}%
}

\newcommand{\lc}{$\lambda$ {\it calculus}\ }

\title[SPLE 2022]% optional, use only with long paper titles
{Software Product Line Engineering 2025}

\subtitle{Simply Typed $\lambda$ Calculus}  % could also be a conference name

\date{TBD}

\author[HSF] % optional, use only with lots of authors
{
  Hafiyyan Sayyid Fadhlillah\\
  \href{mailto:hafiyyan@cs.ui.ac.id}{{\tt hafiyyan@cs.ui.ac.id}}
}
% - Give the names in the same order as they appear in the paper.
% - Use the \inst{?} command only if the authors have different
%   affiliation. See the beamer manual for an example

\institute[
%  {\includegraphics[scale=0.2]{aau_segl}}\\ %insert a company, department or university logo
  Fakultas Ilmu Komputer\\
  Universitas Indonesia
] % optional - is placed in the bottom of the sidebar on every slide
{% is placed on the title page
  Fakultas Ilmu Komputer\\
  Universitas Indonesia

  %there must be an empty line above this line - otherwise some unwanted space is added between the university and the country (I do not know why;( )
}

% specify a logo on the titlepage (you can specify additional logos an include them in
% institute command below
%\pgfdeclareimage[height=1.5cm]{titlepagelogo}{AAUgraphics/aau_logo_new} % placed on the title page
%\pgfdeclareimage[height=1.5cm]{titlepagelogo2}{graphics/aau_logo_new} % placed on the title page
%\titlegraphic{% is placed on the bottom of the title page
%  \pgfuseimage{titlepagelogo}
%%  \hspace{1cm}\pgfuseimage{titlepagelogo2}
%}

\begin{document}
% the titlepage
{\uilogograybg % gray background
%\uilogotitle	% white background
\setbeamercolor{title}{fg=white}	% font color of the presentation title is changable depending on the background color
\setbeamercolor{frametitle}{use=structure,fg=black!80,bg=beamer@barcolor}
\begin{frame}[plain,noframenumbering] % the plain option removes the sidebar and header from the title page
  \titlepage
\end{frame}}
%%%%%%%%%%%%%%%%

\begin{frame}{Agenda}{}
\tableofcontents
\end{frame}
%%%%%%%%%%%%%%%%

\section{Pengantar}

%\begin{frame}{Perubahan Paradigma Berfikir}{}
%
%	\begin{block}{Alvin Toffler, Powershift: Knowledge, Wealth, and Power at the Edge of the 21st Century, 1990 }
%		\pause
%		The illiterate of the 21st century will not be those who cannot read and write, but those who cannot learn, unlearn, and relearn.
%	\end{block}
%
%	\begin{itemize}
%		\pause
%		\item Sebelum mempelajari syntax, semantic dari bahasa pemrograman yang baru, perlu dipelajari dulu paradigma dan filosofi bahasa tersebut
%		\pause
%		\item Untuk dapat mempelajari yang baru kita harus perlu dapat melupakan (mengesampingkan) yang lama dulu. (Minimal untuk sementara)
%	\end{itemize}
%
%\end{frame}

\begin{frame}{$\lambda$ (Lambda) Calculus }{Paparan singkat informal}

  \begin{itemize}
    \item<1-> Kata "{\bf kalkulus}" secara umum berarti sistem baku (formal) untuk aturan inferensi dan aksioma.
    \item<2-> \lc adalah sebuah sistem baku untuk aturan inferensi dan aksioma yang ditandai dengan symbol $\lambda$ untuk memodelkan komputasi berfokus/berbasis \bf{ekspresi}.
    \item<3-> \it{The smallest universal programming language}. Dapat digunakan untuk mensimulasi \textit{turing machine}.
    \item<4-> Diperkenalkan oleh Alonzo Church tahun 1930-an. Pembimbingan doctor dari Alan Turing.
   	\item<5-> Jauh diformulasikan sebelum komputer merupakan barang yang populer.
  \end{itemize}
  \note{
    Berikut ini adalah contoh note di dalam slide.
    Apabila dokumen LaTeX ini sudah diatur dengan benar,
    seharusnya akan ada section khusus menampilkan handout berisikan note ini.
  }
\end{frame}
%%%%%%%%%%%%%%%%

\begin{frame}{$\lambda$ Ekspresi}{}

	\begin{itemize}

		\item Dalam \lc ada tiga jenis ekspresi, yaitu:
		\begin{enumerate}
			\item variabel ($a,b,c,d$)
			\item aplikasi (\it{application})
			\item fungsi (\it{abstraction})
		\end{enumerate}
		\pause
		\item Secara formal didefinisikan rekursif sebagai berikut:
		\begin{tabular}{r c l}
			<ekspresi> & := & <nama> | <fungsi> | <aplikasi> \\
			<fungsi> & := & $\lambda$ <nama>. <ekspresi> \\
			<aplikasi> & := & <ekspresi> <ekspresi>
		\end{tabular}

		\item nama variabel biasanya dinyatakan oleh huruf kecil (a,b,c,...)


	\end{itemize}
\end{frame}

\begin{frame}{Asosiatif dalam aplikasi}
	\begin{enumerate}
	\item Ekspresi bisa diberikan tanda kurung.
	\item jika $E$ adalah ekspresi, maka $(E)$ adalah ekspresi juga.
	\item selain variable dan tanda kurung, simbol lain adalah $lambda$ dan tanda titik {\bf .} (dot).
	\item untuk mengurangi tanda kurung, disepakati asosiatif ke kiri yang artinya penulisan sebagai berikut:
	$$E_1 E_2 E_3 E_4$$
	sama dengan
	$$ (((E_1 E_2) E_3) E_4) $$
	\end{enumerate}

\end{frame}

\begin{frame}{Fungsi}{Sebuah $\lambda$ Ekspresi}

	\begin{enumerate}
		\item<1-> Ekspresi: $$\lambda x. y$$ adalah sebuah fungsi
		\item<2-> variable $x$ setelah simbol $\lambda$ menyatakan argumen untuk fungsi tersebut.
		\item<3-> variable $y$ setelah simbol . ({\it dot}) adalah isi ({\it body}) dari fungsi tersebut.
	\end{enumerate}
\end{frame}

\begin{frame}{Fungsi dalam fungsi}{Sebuah $\lambda$ Ekspresi}

	\begin{itemize}
		\item<1-> di dalam isi sebuah fungsi, bisa berupa ekspresi lagi yang ekspresinya bisa berbentuk salah satu dari tiga bentuk tersebut (variable, fungsi, application)
		\item<2-> Ekspresi: $$\lambda x. \lambda y. z $$ adalah sebuah fungsi.
		\item<3-> Fungsi tersebut diatas memiliki argumen $x$ yang di dalam $isi$ fungsi tersebut, terdapat fungsi lain. Fungsi yang berada di dalam memiliki argument $y$ dan isinya adalah $z$
	\end{itemize}
\end{frame}

\subsection{Terminologi}
\begin{frame}{Terminolog}{Fungsi, $\lambda$ ekspresi,  dan aplikasi fungsi}

	\begin{itemize}
		\item<1-> Ekspresi: $$\lambda x. x$$ adalah sebuah fungsi identitas
		\item<2-> Sebuah ekspresi bisa di-{\bf aplikasi}-kan kepada sebuah ekspresi.
		\item<3-> Ekspresi: $$(\lambda x. x) y$$ adalah aplikasi ekspresi $y$
		(yang dalam hal ini adalah variable) kepada fungsi indentitas tersebut.
		\item<4-> {\bf Aplikasi fungsi} adalah proses meng-substitusi kehadiran variable pada argument $x$ dengan ekspresi $y$ yang artinya setiap kemunculan variable $x$ dalam isi fungsi tersebut akan digantikan oleh variable $y$.
		 $$(\lambda x. x) y \equiv y $$
	\end{itemize}
\end{frame}

\subsection{$\beta$ (beta) Reduction}
\begin{frame}{$\beta$ (beta) Reduction}{\lc}

	\begin{itemize}
		\item Secara umum aplikasi $$(\lambda v. E_1) E_2$$ dievaluasi sebagai $$E_1[v:=E_2]$$
		\item dibaca sebagai kehadiran variabel $v$ didalam $E_1$ digantikan dengan $E_2$.
		\item Aturan ini secara formal disebut {\bf $\beta$ (beta) reduction}
		\item Contoh:
		\[
		\begin{array}{l}
		(\lambda x. (\lambda y. x)) z \\
		\equiv (\lambda y. x)[x:= z] \\
		\equiv (\lambda y. z)\\
		\end{array}
		\]
	\end{itemize}
\end{frame}


\begin{frame}{variabel bebas dan variable terikat}{\lc}

	\begin{itemize}
		\item dalam \lc nama variabel adalah berlaku local didalam definisi tersebut.
		\item dalam ekspresi $$\lambda y. y$$ variabel $y$ dikatakan {\bf terikat}, sementara variable lain yang tidak didahului oleh simbol $\lambda$ dikatakan sebagai variabel {\bf bebas}
		\item Misalkan pada ekspress:
		$$\lambda x. xy$$
		maka variabel $x$ terikat dan variabel $y$ dinyatakan bebas.
		\pause
		\item pada contoh berikut: $$(\lambda x. x)(\lambda y. yx)$$
		perhatikan bahwa variable $x$ di ekspresi sebelah kiri terikat, tapi variabel $x$ di ekspresi sebelah kanan adalah bebas. Kedua variable $x$ tersebut menyatakan hal yang berbeda.

	\end{itemize}
\end{frame}


\begin{frame}{observasi: Fungsi Identitas}{\lc}

	\begin{itemize}
		\item fungsi identitas $$\lambda x. x$$
		\item Contoh evaluasi:
		\[
		\begin{array}{l}
		(\lambda x. x) z \\
		\equiv x[x:= z] \\
		\equiv z\\
		\end{array}
		\]
		\pause
		\item fungsi $$\lambda y. y$$ adalah juga fungsi identitas
		\item Contoh evaluasi:
		\[
		\begin{array}{l}
		(\lambda y. y) z \\
		\equiv y[y:= z] \\
		\equiv z\\
		\end{array}
		\]

	\end{itemize}
	\pause
	\alert{
		secara umum variabel terikat didalam sebuah ekspresi bisa diganti dengan variabel baru lain, asalkan variable baru tersebut tidak muncul bebas dialam ekspresi tersebut.
	}
\end{frame}

\subsection{$\alpha$ (alpha) Conversion}
\begin{frame}{$\alpha$ (alpha) Conversion}{\lc}
	\begin{itemize}
		\item secara general setiap variable bisa digantikan dengan variabel lain dan tetap menyatakan ekspresi yang sama (asalkan dilakukan secara konsisten).
		\item Contoh evaluasi:
		\[
		\begin{array}{l}
		(\lambda x. x) z \\
		\equiv x[x:= z] \\
		\equiv z\\
		\end{array}
		\]

		\item fungsi $$\lambda y. y$$ adalah juga fungsi identitas
		\item Contoh evaluasi:
		\[
		\begin{array}{l}
		(\lambda y. y) z \\
		\equiv y[y:= z] \\
		\equiv z\\
		\end{array}
		\]
		\item secara umum variabel didalam sebuah ekspresi bisa secara konsisten diganti
	\end{itemize}
\end{frame}


\subsection{$\eta$ (eta) Reduction}
\begin{frame}{$\eta$ (eta) Reduction}{\lc}
	\begin{itemize}
		\item Menyatakan sebuah reduksi yang menghilang variable.
		Contoh:

		$$f\ x = g\ x  \equiv f = g$$

		\item Seperti \textit{point free style}
	\end{itemize}
\end{frame}


\section{Evaluasi dan Substitusi}
% motivation for creating this theme
\begin{frame}{Fungsi tidak punya nama}{\lc }

	\begin{itemize}
		\item<1-> Dalam \lc, setiap fungsi tidak diberi nama.
		\item<1-> Dalam mengevaluasinya kita perlu menuliskan keseluruhan fungsi.
		\item<2-> Misalkan fungsi identitas nyatakan equivalen dengan ekspresi $I \equiv (\lambda x.x)$
		\item<2-> Ekspresi $I$ merupakan synonim dengan fungsi identitas tersebut.
		\item<3-> Fungsi identitas $I$ bisa kita aplikasikan dengan fungsi itu sendiri:
		$$ I I \equiv (\lambda x.x) (\lambda x.x)$$
		\item<3-> sebagaimana observasi sebelumnya, untuk menghindari kebingungan dengan adanya variabel yang sama tapi mungkin punya makna berbeda, kita dapat mengubah variable di salah salah ekspresi dengan menerapkan $\alpha-conversion$ menjadi
		$$ I I \equiv (\lambda x.x) (\lambda z.z)$$
		\item<4-> Evaluasi ekspresi tersebut:
		\[
		\begin{array}{l}
		I I \equiv (\lambda x.x) (\lambda z.z)\\
		\equiv x [x:= (\lambda z.z)] \\
		\equiv (\lambda z.z) \\
		\equiv I \\
		\end{array}
		\]
	\end{itemize}
\end{frame}
%%%%%%%%%%%%%%%%
\begin{frame}{Tidak mencampur variabel bebas dan variabel terikat}{\lc }

	\begin{itemize}
		\item<1-> Ketika melakukan substitusi, kita perlu hati-hati agar tidak mencampur antara variable bebas dengan variabel terikat.
		\item<1-> Misalkan ekspresi berikut ini
		$$(\lambda x. (\lambda y. xy)) y$$
		\item<2-> Perhatikan variabel $y$ pada $xy$ adalah terikat, sementara variabel $y$ yang paling kanan adalah bebas.

		\item<3-> Substitusi yang keliru akan menghasilkan:
		\[
		\begin{array}{l}
		(\lambda x. (\lambda y. xy)) y \\
		\equiv (\lambda y. xy) [x:= y] \\
		\equiv (\lambda y.yy) \\
		\end{array}
		\]
		\item<3-> kedua variabel $y$ yang mungkin berbeda,  akhirnya kedua jadi terikat dan harus sama.
		\item<4-> Sebelum di evaluasi, seharusnya bisa diganti dulu variabel terikat nya, yaitu:
		\[
		\begin{array}{l}
		(\lambda x. (\lambda y. xy)) y \\
		\equiv (\lambda x. (\lambda t. xt)) y \\
		\equiv (\lambda t. xt) [x:= y] \\
		\equiv (\lambda t.yt) \\
		\end{array}
		\]

	\end{itemize}
\end{frame}
%%%%%%%%%%%%%%%%

\section{Representasi}

\subsection{Church Numerals}

\begin{frame}{Representasi bilangan cacah}{\lc: Church Numerals}

	\begin{itemize}
		\item<1-> Sebuah bahasa pemrograman perlu dapat merepresentasikan bilangan
		\item<2-> Sebelumnya kita telah pelajari bagaimana representasi bilangan dinyatakan secara simbol (gambar), \item<3-> kemudian ketika kuliah kita mempelajari bahwa bilangan bisa dinyatakan secara binary berdasarkan signal listrik.
		\item<4-> Pada kali ini kita akan pelajari bagaimana {\bf bilangan dinyatakan sebagai fungsi} dalam \lc
	\end{itemize}
\end{frame}


% motivation for creating this theme
\begin{frame}{Representasi bilangan cacah}{\lc: Church's Numerals}

	\begin{itemize}
		\item<1-> $0 \equiv \lambda s. (\lambda z. z)$
		\item<2-> Bila ada symbol $\lambda$ bersarang biasa nya dipersingkat penulisnya menjadi:\\

		$ 0 \equiv \lambda s z. z$
		 \item<3-> $ 1 \equiv \lambda s z. s (z))$
		 \item<4-> $ 2 \equiv \lambda s z. s (s (z)))$
		 \item<5-> $ 3 \equiv \lambda s z. s (s (s (z))))$
		 \item<5-> dan seterusnya
	\end{itemize}


\end{frame}


% motivation for creating this theme
\begin{frame}{Representasi bilangan cacah}{\lc: Church's Numerals}

	\begin{itemize}
		\item<1-> Di \lc ada sebuah fungsi yang disebut fungsi $successor$ dengan simbol $S$.

		\item<2-> Fungsi tersebut dapat memodelkan bilangan secara unary: 1 = S(0), 2=S(S(0)) dan seterusnya.

		\item<3-> Definisinya: $$ S \equiv \lambda w y x. y (wyx) $$
	\end{itemize}

\end{frame}

\subsubsection{Penjumlah}

\begin{frame}{Penjumlahan}{\lc }

	\begin{itemize}
		\item<1-> Misalkan kita ingin menjumlahkan $$2+3$$

		\item<2-> dalam \lc dapat dimodelkan dengan $$2S3$$

		\item<3-> $2S3 \equiv (\lambda sz. s(s(z))) (\lambda w y x. y (wyx)) (\lambda u v. u (u (u (v)))))  $

		\item<4-> apakah bila dievaluasi benar hasilnya 5?
		\item<5-> Mari coba dulu yang lebih sederhana dengan $$0+1$$ $$0S1$$ apakah benar hasilnya 1?
	\end{itemize}

\end{frame}

\subsubsection{Perkalian}

\begin{frame}{Perkalian}{\lc }

	\begin{itemize}
		\item<1-> Perkalian adalah pengulangan penjumlahan dan dapat dimodelkan dengan $$\lambda x y z. x (y z) $$

		\item<2-> Perkalian $$0 x 1$$ dimodelkan dengan $$ (\lambda x y z. x (y z)) 0 1 $$

		\item<2-> Pada penulisan kali ini kita perkenalkan 'literal 0 dan 1' yang kelak akan dikembalikan kepada definisi representasinya di \lc

		\item<3->
		\[
		\begin{array}{lr}
		0*1 & \\

		\pause
		\equiv  (\lambda x y z. x (y z)) 0 1  & $representasi dari $ 0*1 \\

		\pause
		\equiv  (\lambda y z. 0 (y z)) 1 & $aplikasi 0,$ x:=0 \\

		\pause
		\equiv  (\lambda z. 0 (1 z))  & $aplikasi 1,$ y:= 1 \\

		\pause
		\equiv  (\lambda z. (\lambda s. (\lambda t. t)) (1 z))  & $membuka definisi literal $ 0
		\equiv (\lambda s. (\lambda t. t)) \\
		\pause

 		\equiv  (\lambda z. (\lambda s. (\lambda t. t)) (1 z))  & $aplikasi $ (1 z), s:= (1 z) \\

 		\pause
		\equiv  (\lambda z. (\lambda t. t)))  & $definisi $ 0 \\
		\equiv  0 &  \\
		\end{array}
		\]

		\item<4-> Bagaimana dengan perkalian $1 * 2$ (Kita akan berlatih dan kerjakan bersama)?

	\end{itemize}
	\end{frame}

\subsection{Church Boolean}
\subsubsection{True False}
	\begin{frame}{Kondisi Bersyarat (If-then-else)}{\lc: Church Boolean}

			\begin{description}
				\item [$T$] (true) dinyatakan dengan $\lambda x y. x$
				\item [$F$] (false) dinyatakan dengan $\lambda x y. y$
			\end{description}
			\pause
			Contoh: Bagaimana menyatakan {\bf  if $P$ then $E_1$ else $E_2$}
			\pause
			\begin{itemize}
				\item<2-> $P$ adalah sebuah propotional yang hanya dapat berisi benar atau salah.
				\item<3-> Dengan kata lain, $P$ dinyatakan sebuah variable yang bisa berisi literal $T$ atau $F$.
				\item<4-> sehingga peryataan tersebut bisa dituliskan dalam \lc dengan
				$$P E_1 E_2 $$
				\item<5-> Bila $P$ adalah benar ($T$) maka\[
				 \begin{array}{lr}
				P E_1 E_2 & P $ adalah true$ \\
				\equiv (\lambda x y. x ) E_1 E_2 & \beta-reduction \\
				\equiv E_1 & \\
				 \end{array}
				 \]
				\item<5-> Bila $P$ adalah false ($F$) maka\[
				\begin{array}{lr}
				P E_1 E_2  & P $ adalah false$ \\
				\equiv (\lambda x y. y ) E_1 E_2 & \beta-reduction \\
				\equiv E_2 & \\
				\end{array}
				\]

			\end{itemize}
	\end{frame}

\subsubsection{Operasi Logika}
\begin{frame}{Operasi Logika}{\lc: Church's Boolean}

	\begin{description}
		\item [Conjuntion (and)] ($\wedge$) dinyatakan dengan
		$$\wedge \equiv \lambda xy. xy(\lambda uv. v) \equiv  \lambda xy. xyF$$
		\item [Disjunction (or)] ($\vee$) dinyatakan dengan
		$$\vee  \equiv \lambda xy.x(\lambda uv.u)y \equiv \lambda xy.xTy$$
		\item [Negation (not)] ($\neg$) dinyatakan dengan
		$$ \lambda x.x(\lambda uv.v)(\lambda ab.a) \equiv  \lambda x.xFT $$
	\end{description}
	\pause
	\begin{itemize}
		\item Contoh: $\neg T \equiv F$.

		\[
		\begin{array}{lr}
		\neg T & definisi\ \neg  \\
		\equiv (\lambda x.xFT) T  & \beta-reduction \\
		\equiv TFT & definisi\ T \\
		\equiv (\lambda xy. x) F T & \beta-reduction  \\
		\equiv F  &  \\
		\end{array}
		\]

	\end{itemize}
\end{frame}

\begin{frame}{Test zero}{\lc}

	Dalam beberapa bahasa pemrograman akan sangat bermanfaat bila
	memiliki fungsi yang menguji apakah suatu variabel bernilai 0 atau tidak.

	Untuk kebutuhan tersebut misalkan kita definisikan
	$$ Z \equiv  \lambda x. x F \neg F $$

	Untuk memahami cara kerjanya, perhatikan bahwa:
	$$ 0 f a \equiv (\lambda s z. z) f a \equiv a $$
	yang artinya fungsi $f$, diaplikasikan dengan  $a$ sebanyak 0 kali sehingga tetap a.

	\pause
	Perhatikan juga bahwa bila fungsi $F$ diaplikasikan dengan apapun, maka akan menghasilkan fungsi identitas $I$.
	$$Fa \equiv (\lambda x y . y ) a \equiv \lambda y.y \equiv I$$

	Sekarang kita perhatikan bagaimana cara kerja fungsi $Z$ ini. Perhatikan fungsi $Z$ diaplikasi dengan 0.
	\[
	\begin{array}{rlr}
    Z0 \equiv & (\lambda x.xF \neg F)0   & $definisi Z dan $ \beta-reduction \\
         \equiv & 0F \neg F & $karena F diaplikasi dengan $ \neg F $ sebanyak 0 kali$ \\
         \equiv &  \neg F &  \\
         \equiv & T   & \\
    \end{array}
	\]

\end{frame}

\begin{frame}{Test zero}{\lc}
	Representasi {\it church numeral}, menyatakan bilangan sebagai enumerasi yang merupakan fungsi menerima fungsi dan argumen kemudian menerapkan fungsi tersebut terus menerus sebanyak enumerasinya.

	Ingat bahwa bila fungsi $F$ diaplikasikan dengan apapun, maka akan menghasilkan fungsi identitas $I$.
	$$Fa \equiv (\lambda x y . y ) a \equiv \lambda y.y \equiv I$$

	Perhatikan fungsi $Z$ diaplikasi dengan bilangan bukan 0, misalkan N.
	\[
	\begin{array}{rlr}
	ZN \equiv & (\lambda x.xF \neg F)N & $definisi Z dan $ \beta-reduction \\
	\equiv & NF \neg F & $karena F diaplikasi dengan $ \neg F $ sebanyak N kali, hasilnya I $ \\
	\equiv &  I F &  \\
	\equiv & F  & \\
	\end{array}
	\]

\end{frame}

\begin{frame}{Predecessor (Nilai sebelumnya)}{\lc}
	Untuk menghitung nilai sebelumnya, digunakan representasi pair $(n, n-1)$ elemen kedua dari pair dinyatakan sebagai $predessor$.  Sebuah $pair(a,b)$ dinyatakan dalam \lc dengan
	$$ (a,b) \equiv \lambda z. z a b$$

	Untuk mengambil elemen pertama, fungsi pair tersebut di aplikasikan dengan $T$.
	$$ first (a,b) \equiv (\lambda z. z a b) T \equiv Tab \equiv a $$

	dan untuk mengambil elemen kedua, fungsi pair tersebut diaplikasikan dengan $F$.
	$$ second (a,b) \equiv (\lambda z. z a b) F \equiv Fab \equiv b $$

	Untuk dapat menggenerate pair ke-n, didefinisikan fungsi generator: $$\Phi \equiv (\lambda pz. z(S(pT))(pT)) $$

	Fungsi predecessor didefinisikan dengan sebelumnya menggenerate pair ke-n, kemudian mengambil elemen kedua. Generator akan menerapkan sebanyak n kali dari inisialisasi awal yaitu  $\lambda z.z00$. Perhatikan bahwa kita memodelkan sehingga dianggap predecessor dari 0 adalah 0. Dengan acuan tersebut didefinisikan:
	$$ P \equiv (\lambda n. n\ \Phi (\lambda z.z00)F )$$

\end{frame}

\subsubsection{Equality dan inequalities}
\begin{frame}{Equality dan inequalities}{\lc}

Dengan memanfaatkan definisi predecessor $P$ kita bisa definisikan "predikat lebih besar sama dengan" ($\geq $)  kita simbolkan dengan $G$ sebagai berikut:
$$ G \equiv (\lambda xy.Z(xPy))$$

Ini artinya kita memeriksa, apakah bila fungsi predecessor $P$ diaplikasikan dengan $y$ sebanyak $x$ kali akan menghasilkan $0$ atau tidak. Bila $x$ lebih besar atau sama dengan $y$ maka hasilnya akan 0, karena $y$ akan dikurangi satu kali terus menerus sebanyak $x$ kali. Berlaku juga sebaliknya.
Bila hasilnya 0, maka fungsi penguji $Z$ akan memberikan jawaban $T$ (true).

Untuk mendefinisikan persamaan ($equality$) dari bilangan didefinisikan $E$:
$$ E \equiv (\lambda xy. \wedge (Z(xPy))(Z(yPx))) $$

yang menyatakan bahwa bila $x\geq y$ dan $y \geq x$ maka $x=y$

\end{frame}

\subsection{Rekursif}
\begin{frame}{Rekursif}{\lc}

	Definisi rekursif dimodelkan dalam \lc dengan fungsi yang disebut $Y$ dan fungsi tersebut me-regenerate fungsi kembali. Didefinisikan sebagai berikut:
	$$ Y \equiv  (\lambda y.(\lambda x.y(xx))(\lambda x.y(xx))) $$

	Bagaimana penerapan-nya? Misalkan ada sebuah fungsi rekursif $R$.
	\[
	\begin{array}{rlr}
	YR \equiv  &  ((\lambda x.R(xx))(\lambda x.R(xx)))  &  definisi \\
	     \equiv  & R((\lambda x.R(xx))(\lambda x.R(xx)))) & \beta-reduction \\
	     \equiv  & R(YR) & rekursif \\
	 \end{array}
	\]
	Dengan kata lain, bila fungsi $Y$ diaplikasikan dengan sebuah fungsi $R$, maka akan menghasilkan pemanggilan rekursi dari fungsi $R$ tersebut.

	Contoh:
	$$ \Sigma_{i=0}^n i = n + \Sigma_{i=0}^{n-1} i  $$
\end{frame}

\begin{frame}{Rekursif}{\lc}


	$$ Y \equiv  (\lambda y.(\lambda x.y(xx))(\lambda x.y(xx))) $$

	Contoh:
	$$ \Sigma_{i=0}^n i = n + \Sigma_{i=0}^{n-1} i  $$
	\pause

	Fungsi rekursif tersebut diberi nama $R$ dan dimodelkan dalam \lc menjadi:
	$$ R \equiv  (\lambda rn.Zn0(nS(r(Pn)))) $$

	\pause
	Misalkan dengan parameter $n=3$, penerapan rekursifnya menjadi:

	$$YR3 = R(YR)3 = Z30(3S(YR(P3)))$$
	\pause
	Karena 3 tidak sama dengan 0, maka menjadi:
	$$3S(YR2)$$
	\pause
	Bila kembali diterapkan definisi $Y$, maka akan didapat:
	$$3S(YR2) \equiv 3S(2S(YR1)) \equiv 3S(2S(1S0)) \equiv 6 $$



\end{frame}

%\section{Aturan lain}
%
%	\begin{frame}{Aturan lain}{\lc }
%		\begin{block}{Beberapa aturan lain dalam \lc}
%		\begin{description}
%			\item [$\eta$ (eta) reduction], menghilangkan $\lambda$
%			$$ (\lambda x. f x) \equiv f$$
%			\item [$\lambda$ (lambda) abstraction], menambahkan $\lambda$
%			$$ f \equiv (\lambda x. f x)$$
%
%		\end{description}
%		\end{block}
%	\end{frame}

\section{Type System}

	\begin{frame}{Motivasi: Simply Typed Lambda Calculus }{}
		\begin{block}{Sebuah Paradox}
			Perhatikan kalimat berikut:
			\pause

			{\LARGE {\center
				\alert{Saya seorang pembohong}
			}}
		\end{block}
	\end{frame}


	\begin{frame}{Russel's Paradox}{Motivasi: Simply Typed Lambda Calculus }
		{\it Barber Paradox} \vspace{1cm}

		{\Large  A barber (who is a man) shaves all and only those men who do not shave themselves.

			\vspace{1cm}
			Does he shave himself?}

		\pause
		\vspace{1cm}
		{\it Na\"ive Comprehension (NC)}

		$$(NC) \exists A . \forall . x (x \in A \equiv  \Phi )$$

		where $A$ is not free in the formula $\Phi$. This says, “There is a set $A$ such that for any object $x$, $x$ is an element of $A$ if and only if the condition expressed by $\Phi$ holds.” Russell’s paradox arises by taking $\Phi$  to be the formula: $x \notin x$.

	\end{frame}

	\note{Na\"ive Comprehension in Contemporary Logic. When set theory is na\"ive.
		 https://plato.stanford.edu/entries/russell-paradox/  }


\subsection{Abstract Syntax}
\begin{frame}{Abstract Syntax}{Simple Type \lc }

	\[
	\begin{array}{rcl}
	T, U &\in& \mathbf{Type} \\
	T, U &::=& T \to U \\
	&& \cdots\\
	x,y,z &\in& \mathbf{Var} \\
	t,u &\in& \mathbf{Term} \\
	t,u &::=& x \\
	&\mid& t\,u \\
	&\mid& \lambda {x:T}\cdot t \\
	&& \cdots
	\end{array}
	\]

	Simbol $\dots$ menyatakan bagian dari bahasa atau tipe yang relatif sudah kita
	kenal dan tidak membutuhkan $pure-types$.

	Pada Simply Type $\lambda$ Calculus, tipe dinyatakan sebagai sebuah elemen, dalam paparan kali ini dengan huruf $T$, dan $U$. Penulisan $x:T$ dibaca sebagai sebuah term atau variable $x$ bertipe $T$. Tipe dapat juga berupa aplikasi fungsi diantara dua tipe $T \to U$.

	Berbeda dengan asosiatif pada fungsi aplikasi, pada penulisan tipe asosiatif yang digunakan adalah asosiatif ke kanan.

\end{frame}

\subsection{Type Judgement}
\begin{frame}{Type Judgement}{Simple Type \lc }

	Dengan adanya penerapan sistem tipe ({\it type system}) sebuah prosedur pemeriksaan
	tipe akan dibutuhkan. Prosedur tersebut di literatur disebut dengan istilah
	\emph{type judgment}.
%	Terjemahan langsung dalam bahasa indonesia akan
%	berarti: \emph{Pengadilan Jenis}. Untuk menghidari kebingungan kita akan tetap
%	gunakan istilah \emph{type judgment} dan beberapa istilah terkait lain nya seperti $term$ dan $type$.

	Prosedur \emph{type judgment} memiliki tiga komponen yaitu:

	\begin{itemize}
		\item  \emph{type environment}, berisi semacam $dictionary$ yang memetakan
		variable yang berada dalam $term$ ke $type$.
		\item $term$ yang perlu ditentukan jenis nya ($type$)
		\item usulan $type$ untuk $term$
	\end{itemize}

    \emph{Type Judgment} biasaya ditulis secara formal sebagai berikut:
    $$  \Gamma \vdash t : T$$

    dengan $\Gamma$ adalah \emph{type environment}, $t$ adalah $term$ dan $T$ adalah  $type$.

    Penulisan $\Gamma, x : T$ artinya, {\it type environment} menyatakan
    bahwa
    $x$ bertipe $T$ dan tipe dari variable lain masih berada dalam $\Gamma$.

\end{frame}


\subsection{Inference System}
\begin{frame}{Inference System}{Simple Type \lc }

	Dalam \emph{type system} untuk melakukan validitas dari tipe sesuai dengan \emph{type judgment}, dibuatlah \emph{inference system} yang biasanya terdiri dari:


	\begin{itemize}
		\item Sebuah  \emph{axiom} yang berisi sebuah \emph{type judgment}, yang biasanya buat garis lurus diatasnya.
		\item sebuah \emph{inference rule} terdiri dari sebuah \emph{type judgment} (yang menyatakan kesimpulan), yang diberi garis diatasnya. Diatas garis bisa berada satu atau lebih \emph{type judgments } yang disebut  \emph{premises}).
		Sebuah axiom bisa dlihat sebagai \emph{inference rule} tanpa premises.
	\end{itemize}

\end{frame}


\begin{frame}{Inference System}{Simple Type \lc }

	\begin{gather}
	\begin{gathered}[b]
	\hline
	\Gamma, x : T \vdash  x :T
	\end{gathered} \\
	\begin{gathered}[b]
	\Gamma \vdash t : U \to T \qquad
	\Gamma \vdash u : U \\
	\hline
	\Gamma \vdash tu : T
	\end{gathered} \label{stlc:Tapp}\\
	\begin{gathered}[b]
	\Gamma, x:U \vdash t : T\\
	\hline
	\Gamma \vdash (\lambda {x : U} \cdot t) : U \to T
	\end{gathered} \\
	\intertext{Untuk lambda calculus yang sudah berikan ekstension literal terkait numeric aritmatika (dalam contoh ini simbol operator $+$ dan tipe $Num$ ) akan memiliki tambahan aturan:}
	\begin{gathered}[b]
	\hline
	\Gamma \vdash c : \mathbf{Num}
	\end{gathered} \\
	\begin{gathered}[b]
	\Gamma \vdash t : \mathbf{Num} \qquad
	\Gamma \vdash u : \mathbf{Num} \\
	\hline
	\Gamma \vdash t+u : \mathbf{Num}
	\end{gathered}
	\end{gather}

Sebuah penetapan tipe yang tidak bisa diturunkan menggunakan sistem tersebut biasa disebut sebagai \emph{ill-typed}.


\end{frame}

\begin{frame}{Contoh: type inferencing}{SImple Type \lc }


\begingroup\small%
\[
\mathit{avg} : \mathbf{Num} \to\mathbf{Num} \to\mathbf{Num} \vdash
(\lambda f : \mathbf{Num} \to \mathbf{Num} \cdot f\,3) (\mathit{avg}\,2)
: \mathbf{Num}
\]
\endgroup%

 \begingroup\tiny%
 \[
 \begin{gathered}
 \begin{gathered}[b]
 \begin{gathered}[b]
 \begin{gathered}[b]
 \hline
 \mathit{avg} : \cdots,
 f : \beta
 \vdash
 f
 :  \delta \to \alpha
 \end{gathered}
 \qquad
 \begin{gathered}[b]
 \hline
 \mathit{avg} : \cdots,
 f : \cdots
 \vdash
 3
 : \delta
 \end{gathered}
 \\
 \hline
 \mathit{avg} : \cdots,
 f : \beta
 \vdash
 f\,3
 : \alpha
 \end{gathered}\\
 \hline
 \mathit{avg} : \cdots
 \vdash
 (\lambda f : \beta \cdot f\,3)
 : \beta \to \alpha
 \end{gathered}
 \qquad
 \begin{gathered}[b]
 \begin{gathered}[b]
 \hline
 \mathit{avg} : \mathbf{Num} \to\mathbf{Num} \to\mathbf{Num} \vdash
 \mathit{avg}
 :  \gamma \to \beta
 \end{gathered}
 \qquad
 \begin{gathered}[b]
 \hline
 \mathit{avg} : \cdots
 \vdash
 2
 : \gamma
 \end{gathered}
 \\
 \hline
 \mathit{avg} :  \mathbf{Num} \to\mathbf{Num} \to\mathbf{Num}
 \vdash
 \mathit{avg}\,2
 : \beta
 \end{gathered}
 \\
 \hline
 \mathit{avg} : \mathbf{Num} \to\mathbf{Num} \to\mathbf{Num} \vdash
 (\lambda f : \beta \cdot f\,3) (\mathit{avg}\,2)
 : \alpha
 \end{gathered}
 \]
 \endgroup%

\end{frame}


\begin{frame}{Contoh: type inferencing (lanj.)}{Simple Type \lc }
	  \begingroup\tiny%
	  \[
	  \begin{gathered}
	  \begin{gathered}[b]
	  \begin{gathered}[b]
	  \begin{gathered}[b]
	  \hline
	  \mathit{avg} : \cdots,
	  f : \mathbf{Num} \to \mathbf{Num}
	  \vdash
	  f
	  :  \mathbf{Num} \to \mathbf{Num}
	  \end{gathered}
	  \qquad
	  \begin{gathered}[b]
	  \hline
	  \mathit{avg} : \cdots,
	  f : \cdots
	  \vdash
	  3
	  : \mathbf{Num}
	  \end{gathered}
	  \\
	  \hline
	  \mathit{avg} : \cdots,
	  f : \mathbf{Num} \to \mathbf{Num}
	  \vdash
	  f\,3
	  : \mathbf{Num}
	  \end{gathered}\\
	  \hline
	  \mathit{avg} : \cdots
	  \vdash
	  (\lambda f : \mathbf{Num} \to \mathbf{Num} \cdot f\,3)
	  : \mathbf{Num} \to \mathbf{Num}
	  \end{gathered}
	  \qquad
	  \begin{gathered}[b]
	  \begin{gathered}[b]
	  \hline
	  \mathit{avg} : \mathbf{Num} \to\mathbf{Num} \to\mathbf{Num} \vdash
	  \mathit{avg}
	  :  \mathbf{Num} \to\mathbf{Num} \to\mathbf{Num}
	  \end{gathered}
	  \qquad
	  \begin{gathered}[b]
	  \hline
	  \mathit{avg} : \cdots
	  \vdash
	  2
	  : \mathbf{Num}
	  \end{gathered}
	  \\
	  \hline
	  \mathit{avg} :  \mathbf{Num} \to\mathbf{Num} \to\mathbf{Num}
	  \vdash
	  \mathit{avg}\,2
	  : \mathbf{Num}
	  \end{gathered}
	  \\
	  \hline
	  \mathit{avg} : \mathbf{Num} \to\mathbf{Num} \to\mathbf{Num} \vdash
	  (\lambda f : \mathbf{Num} \to \mathbf{Num} \cdot f\,3) (\mathit{avg}\,2)
	  : \mathbf{Num}
	  \end{gathered}
	  \]
	  \endgroup%

	Bentuk penulisan seperti ini biasa disebut \emph{type inference tree}.

\end{frame}


\begin{frame}{Contoh:  ill typed}{Simple Type \lc }

	Ada kasus ketika \emph{type inference tree} tidak sampai pada aksiom. Untuk kasus tersebut kita menyatakan bahwa type yang hendak diuji sebagai \emph{ill typed}. Perhatikan bagian kanan tidak mencapai aksiom, ditandai dengan tidak ada garis diatas nya.

	  \begingroup\tiny%
	  \[
	  \begin{gathered}
	  \begin{gathered}[b]
	  \begin{gathered}[b]
	  \begin{gathered}[b]
	  \hline
	  \mathit{avg} : \cdots,
	  f : \mathbf{Num} \to \mathbf{Num}
	  \vdash
	  f
	  :  \mathbf{Num} \to \mathbf{Num}
	  \end{gathered}
	  \qquad
	  \begin{gathered}[b]
	  \hline
	  \mathit{avg} : \cdots,
	  f : \cdots
	  \vdash
	  3
	  : \mathbf{Num}
	  \end{gathered}
	  \\
	  \hline
	  \mathit{avg} : \cdots,
	  f : \mathbf{Num} \to \mathbf{Num}
	  \vdash
	  f\,3
	  : \mathbf{Num}
	  \end{gathered}\\
	  \hline
	  \mathit{avg} : \cdots
	  \vdash
	  (\lambda f : \mathbf{Num} \to \mathbf{Num} \cdot f\,3)
	  : \mathbf{Num} \to \mathbf{Num}
	  \end{gathered}
	  \qquad
	  \begin{gathered}[b]
	  \mathit{avg} :  \mathbf{Num} \to\mathbf{Num} \to\mathbf{Num}
	  \vdash
	  \mathit{avg}
	  : \mathbf{Num}
	  \end{gathered}
	  \\
	  \hline
	  \mathit{avg} : \mathbf{Num} \to\mathbf{Num} \to\mathbf{Num} \vdash
	  (\lambda f : \mathbf{Num} \to \mathbf{Num} \cdot f\,3) \mathit{avg}
	  : \mathbf{Num}
	  \end{gathered}
	  \]
	  \endgroup%



\end{frame}

\begin{frame}{Contoh: mengusul kan tipe baru}{Simple Type \lc }

	Kadang kita belum tahu semua kemungkinan tipe, kita dapat mengusulkan perkiraan tipe terlebih dahulu kemudian menguji apakah usulan tersebut dalam membentuk susunan yang valid atau \emph{ill typed}. Biasanya menggunakan huruf kecil yunani. Dalam contoh ini kita akan dapati: $\alpha = \gamma = \delta = \mathbf{Num}$

	\begingroup\tiny%
	\[
	\begin{gathered}
	\begin{gathered}[b]
	\begin{gathered}[b]
	\begin{gathered}[b]
	\hline
	\mathit{avg} : \cdots,
	f : \beta
	\vdash
	f
	:  \delta \to \alpha
	\end{gathered}
	\qquad
	\begin{gathered}[b]
	\hline
	\mathit{avg} : \cdots,
	f : \cdots
	\vdash
	3
	: \delta
	\end{gathered}
	\\
	\hline
	\mathit{avg} : \cdots,
	f : \beta
	\vdash
	f\,3
	: \alpha
	\end{gathered}\\
	\hline
	\mathit{avg} : \cdots
	\vdash
	(\lambda f : \beta \cdot f\,3)
	: \beta \to \alpha
	\end{gathered}
	\qquad
	\begin{gathered}[b]
	\begin{gathered}[b]
	\hline
	\mathit{avg} : \mathbf{Num} \to\mathbf{Num} \to\mathbf{Num} \vdash
	\mathit{avg}
	:  \gamma \to \beta
	\end{gathered}
	\qquad
	\begin{gathered}[b]
	\hline
	\mathit{avg} : \cdots
	\vdash
	2
	: \gamma
	\end{gathered}
	\\
	\hline
	\mathit{avg} :  \mathbf{Num} \to\mathbf{Num} \to\mathbf{Num}
	\vdash
	\mathit{avg}\,2
	: \beta
	\end{gathered}
	\\
	\hline
	\mathit{avg} : \mathbf{Num} \to\mathbf{Num} \to\mathbf{Num} \vdash
	(\lambda f : \beta \cdot f\,3) (\mathit{avg}\,2)
	: \alpha
	\end{gathered}
	\]
	\endgroup%

\end{frame}

\section{Latihan}
\begin{frame}{Latihan: $\beta$-reduction }{\lc}

	\begin{enumerate}
		\item<1-> $(\lambda x. (x y)) (\lambda z. z)$
		\item<2-> $(\lambda x. (\lambda y. x y) x) (\lambda z. w) $
		\item<3-> $(\lambda f.  (\lambda g. (f f) g) (\lambda h. k h) ) (\lambda x y. y)$
		\item<4-> $((\lambda z. z) (\lambda y. y y)) ((\lambda x. x) a)$
		\item<5-> $(\lambda x y. x y) y z	$
	\end{enumerate}


\end{frame}

\begin{frame}{Latihan: Type System}{\lc}

	Buktikan bahwa tipe berikut ini benar atau tidak.


	\begin{enumerate}
	    \item<1->  $(\lambda (x:int). (\lambda (y:int). x + y))$
		\item<2-> $(\lambda (f:((int\to string)\to int\to string)) (g:(int\to string)) (x:int). ((f g) x))$
		\item<3-> $(\lambda (f:(int\to int\to string)) (g:(int\to int)) (x:int) (y:int). ((f x) (g y)))$
		\item<4-> $(\lambda (f:((string\to int\to string)\to int\to string)) (x:(int\to string)) (a:int) (y:(string\to int\to string)). (y (x a)))$
		\item<5-> $(\lambda (x:(int\to int)) (y:int). x x (y + 3))$
	\end{enumerate}

\end{frame}


\end{document}


%%%%%
%%%## template
\begin{frame}{template}{}

	\begin{enumerate}
		\item<1-> Permasalahan dengan GOTO \textit{statement} (diakui, sudah ditinggalkan)
		\item<2-> Permasalahan dengan \textbf{pointers}  (diakui, mulai ditinggalkan)
		\item<3-> \alert{Permasalahan dengan \textbf{STATE}} (diakui, belum ditinggalkan)
		\item<4-> Permasalahan dengan OO (masih diteliti, diperdebatkan)
	\end{enumerate}
	\alert{Baca: http://blog.cleancoder.com}
	\begin{itemize}
		\item Berawal dari tulisan Edsger Dijkstra. Makalah klasik: “Go To Statement Considered Harmful”
		dan beberapa makalah.
		\item Dijkstra menunjukkan permasalah dengan GOTO dan mengganti dengan \textit{control flow }seperti if/then/else and while loops.
		\item Setiap perintah GOTO bisa digantikan dengan if/then/else dan loop, sehingga membuat pembacaan dan perancangan program menjadi lebih mudah dan terstrukturu.
		\item Berkembangnya bahasa struktural, seperti C, Pascal.
	\end{itemize}
\end{frame}
