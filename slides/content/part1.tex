% TODO L01 INTRODUCTION

\ifuniversity{recording}{\date{April 19, 2023}\setpicture[250]{apr21-o25a}\setcopyright{}}
\ifuniversity{ulm}{\date{April 18, 2023}\setpicture[250]{apr21-o25a}}
\ifuniversity{magdeburg}{\setpicture[70]{ovgu-autumn1}\setcopyright{Photo: Hannah Theile (OVGU)}}
\ifuniversity{bern}{\setpicture[0]{unibe-main-building}}
\ifuniversity{paderborn}{\date{April 16, 2024}\setpicture[25]{zm2}}
\ifuniversity{braunschweig}{\date{October 16, 2024}}
\ifuniversity{indonesia}{\date{February 3, 2025}}

\author{Thomas Thüm, Timo Kehrer, Elias Kuiter}
\lecture{Introduction}{introduction}

\inputuniversity{content/01-prelude}

\section{Introduction to Product Lines}
\subsection{Handcrafting and Customization}
\begin{frame}{What do these examples have in common?}
	\begin{fancycolumns}[columns=3,widths={35,28},animation=none]
		\pic[width=\linewidth]{wedding}
		
		\pic[width=\linewidth]{shoes}
	\nextcolumn
		\pic[width=\linewidth,trim=85 0 0 0]{eiffel-tower}
	\nextcolumn
		\pic[width=\linewidth,trim=70 35 0 0]{draisine}
		
		~
		\uncover<2->{\begin{definition}{Customization}
			\begin{itemize}
			\item aka.\ handcrafting
			\item labor-intensive production
			\item highly individual goods
			\end{itemize}
		\end{definition}}
	\end{fancycolumns}
\end{frame}

\begin{frame}{Customization of Elevators}
	\begin{fancycolumns}[columns=4,T]
		\centering\pic[width=\linewidth]{elevator1-out}

		two buttons
	\nextcolumn
		\centering\pic[width=\linewidth]{elevator2-out2}

		one button
	\nextcolumn
		\centering\pic[width=\linewidth]{elevator3-out}

		keyhole
	\nextcolumn
		\centering\pic[width=\linewidth]{elevator2-out1}

		floor display
	\end{fancycolumns}
\end{frame}

\begin{frame}{Customization of Elevators}
	\begin{fancycolumns}[columns=4,widths={28,21,28,21},T]
		\centering\pic[width=\linewidth]{elevator1-in2}

		no button to close door
	\nextcolumn
		\centering\pic[width=\linewidth]{elevator3-in1}

		two keyholes
	\nextcolumn
		\centering\pic[width=\linewidth]{elevator4-in}

		keycard
	\nextcolumn
		\centering\pic[width=\linewidth]{elevator2-in2}

		double tap for~undo
	\end{fancycolumns}
\end{frame}

\subsection{Mass Production}
\begin{frame}[b]{\myframetitle}
	\vspace{-\textheightoftitle}
	\begin{fancycolumns}[b]
		\begin{definition}{Mass Production \mysource{\fospl\mypages{3--4}}}
			\begin{itemize}
			\item consequence of industrialization
			\item goods are produced from standardized parts
			\item improved productivity wrt.\ handcrafting
			\item reduced costs, improved quality
			\item but: (almost) no individualism
			\end{itemize}
		\end{definition}
		\uncover<2->{\begin{example}{Principle: One Size Fits All}
			\begin{itemize}
			\item t-shirts: XS, S, M, L, XL, XXL
			\item swiss-army knife 
			\end{itemize}
		\end{example}
		\centering\picDark[width=.45\linewidth]{wollmilchsau}}
	\nextcolumn
		\pic[width=\linewidth]{swiss-army-knife}
		\uncover<3->{\begin{note}{Mass Production for Software?\mysource{\fospl\mypage{7}}}
			\mycite{The idea is to provide software that satisfies the needs of most customers, which leads almost automatically to the situation, in which customers miss desired functionality and are overwhelmed with functionality they do not need actually (just think of any contemporary office or graphics program). It is often this generality that makes software complex, slow, and buggy.}
		\end{note}}
	\end{fancycolumns}
\end{frame}
% industrial revolution/era:
% - John Hall, exchangable parts 1826, 25 years of trials (source?)
% - Henry Ford/Ransom Olds, production/assembly line, 1901 (source?)
% - 1961 first industrial roboter at General Motors
% - 1980s automatic assembly lines

\subsection{Mass Customization}
\begin{frame}{About Every Second Car is Unique}
	\centering\pic[width=.66\linewidth]{many-cars}
\end{frame}
\begin{frame}[label=MassCustomization]{\myframetitle}
	\begin{fancycolumns}[widths={45}]
		\begin{definition}{Mass Customization\mysource{\fospl\mypage{4}}}
			\begin{itemize}
			\item = mass production + customization
			\item customized, individual goods at costs similar to mass production
			\end{itemize}
		\end{definition}
		\begin{exampletight}{Car Configuration}
			\pic[width=\linewidth]{toyota-aygo-wheels}
		\end{exampletight}
	\nextcolumn
		\begin{exampletight}{Car Production}
			\pic[width=\linewidth,trim=0 50 0 240,clip]{car-manufacturing}
		\end{exampletight}
		\begin{example}{Other Domains}
			bikes, computers, electronics, tools, medicine, clothing, food, financial services, \ldots, software?
		\end{example}
	\end{fancycolumns}
\end{frame}

\begin{frame}{Mass Customization for Software?}
	\begin{fancycolumns}[b,widths={55},animation=none]
		\begin{definition}{Mass Customization for Software?}
			\begin{itemize}
			\item customization: individual software developed using Waterfall model or Scrum
			\item mass production: standard software developed once for millions or billions of users (e.g.,~Whatsapp messenger)
			\item mass customization: software product lines
			\end{itemize}
		\end{definition}
		\uncover<2->{\begin{note}{Why Software Product Lines?}
			\begin{itemize}
			\item resource limitations: memory, performance, energy
			\item different hardware
			\item different laws
			\item goal: avoid expensive customization
			\item how is software developed?
			\end{itemize}
		\end{note}}
	\nextcolumn
		\pic[width=\linewidth,trim=250 0 0 0,clip]{car-tower}
	\end{fancycolumns}
\end{frame}

\subsection{Recap: The Software Life Cycle}
\begin{frame}{The Project Cartoon}
	\renewcommand{\projectcartoonwidth}{.135}\only<9|handout:2>{\addtocounter{framenumber}{+1}}
	\uncover<2-|handout:1-2>{\alt<-8|handout:1>{\hprojectcartoon{01}{how the customer explained it}}{\alt<9|handout:2>{\hprojectcartoon{01}{Requirements}}{\projectcartoon{01}{Requirements}}}}% requirements
	\uncover<3->{\alt<-8,10-|handout:1>{\hprojectcartoon{02}{how the project leader understood it}}{\projectcartoon{02}{how the project leader understood it}}}% modeling
	\uncover<4->{\alt<-8,10-|handout:1>{\hprojectcartoon{03}{how the analyst designed it}}{\projectcartoon{03}{how the analyst designed it}}}% architecture and design
	\uncover<5->{\alt<-8,10-|handout:1>{\hprojectcartoon{04}{how the programmer implemented it}}{\projectcartoon{04}{how the programmer implemented it}}}% implementation
	\uncover<6->{\alt<-8,10-|handout:1>{\hprojectcartoon{05}{what the beta testers received}}{\projectcartoon{05}{what the beta testers received}}}% testing
	\uncover<7->{\alt<-8,10-|handout:1>{\hprojectcartoon{10}{how it was supported}}{\projectcartoon{10}{how it was supported}}}% maintenance
	\uncover<8->{\alt<-8|handout:1>{\hprojectcartoon{13}{what the customer really needed}}{\alt<9>{\hprojectcartoon{13}{Product}}{\projectcartoon{13}{Product}}}}% customer / SE II
	\\
\end{frame}

\begin{frame}
	\waterfallcartoon\\
\end{frame}

\subsection{Features and Products of a Domain}
\begin{frame}{What is a Feature?}
	\begin{fancycolumns}[widths={45},t]
		\uncover<2->{\begin{definition}{Feature \deutsch{Feature} \mysource{\fospl\mypage{18}}}
			\mycite{A \emph{feature} is a characteristic or end-user-visible behavior of a software system.}
		\end{definition}}
		\centering\xkcd{2369}{width=.9\linewidth,trim=35 35 35 35,clip}
	\nextcolumn
		\uncover<3->{\begin{note}{Feature in a Product Line \mysource{\fospl\mypage{18}}}
			\mycitebegin Features are used in product-line engineering
			\begin{itemize}
				\item to specify and communicate commonalities and differences of the products between stakeholders \deutsch{Akteure} and
				\item to guide structure, reuse, and variation across all phases of the software life cycle.\myciteend
			\end{itemize}
		\end{note}}
		\renewcommand{\projectcartoonwidth}{.18}\scriptsize
		\uncover<4->{\waterfallcartoon}\\
	\end{fancycolumns}
\end{frame}
% TODO picture of a program highlighting one or two features? settings? changelog? marketing description of new features (Github?)?

% goals of features
%a distinctively identifiable functional abstraction that must be implemented, tested, delivered, and maintained” (Kang et al.
%a product characteristic from user or customer views, which essentially consists of a cohesive set of individual requirements” (Chen et al.
%an optional or incremental unit of Zave 2003) 

\begin{frame}{What is a Product?}
	\begin{fancycolumns}[t,animation=none]
		\begin{definition}{Product \mysource{\fospl\mypage{19}}}
			\mycite{A \emph{product of a product line} is specified by a valid feature selection (a subset of the features of the product line). A feature selection is \emph{valid} if and only if it fulfills all feature dependencies.}
		\end{definition}
		\uncover<2->{\begin{note}{Note on Terminology}
			\begin{itemize}
				\item in this course:\\product = product variant = variant
				\item software product: a product consisting only of software
				\item software is more than a program: requirements, models, source code, tests, documentation
				\item this course focuses on source code
			\end{itemize}
		\end{note}}
		% TODO exemplify difference between hardware product, software product (VLC?), and program (reuse pictures from before)
	\nextcolumn
		\uncover<3->{
			\begin{exampletight}{Product Map for Eclipse (excerpt)}
				\picDark[width=\linewidth,trim=0 460 635 0,clip]{eclipse-product-map}
			\end{exampletight}
		}
	\end{fancycolumns}
\end{frame}
% TODO Venn diagram could also be a good visualization: products correspond to sets and features to set elements

\begin{frame}{What is a Domain?}
	\begin{fancycolumns}[animation=none]
		\begin{definition}{Domain  \mysource{\fospl\mypage{19}}} % TODO cite CE00 here too?
			\mycitebegin A \emph{domain} is an area of knowledge that:
			\begin{itemize}
				\item is scoped to maximize the satisfaction of the requirements of its stakeholders,
				\item includes a set of concepts and terminology understood by practitioners in that area,
				\item and includes the knowledge of how to build software systems (or parts of
				software systems) in that area.\myciteend
			\end{itemize}
		\end{definition}
		\uncover<2->{\begin{note}{Features of a Domain}
			\begin{itemize}
				\item a feature is a domain abstraction
				\item identification of features in a domain requires domain expertise
				\item later: select features for a product line?
			\end{itemize}
		\end{note}}
	\nextcolumn
		\pic[width=.5\linewidth]{many-cars}

		\hfill\xkcd{2369}{width=.5\linewidth,trim=35 35 35 35,clip}\hfill{}

		\hfill\pic[width=.5\linewidth]{eclipse-luna}
	\end{fancycolumns}
\end{frame}
% TODO add more examples for domains

\subsection{Software Product Line}
\begin{frame}[label=SoftwareProductLine]{\myframetitle}
	\begin{fancycolumns}[widths={78},columns=1]
		\begin{definition}{Software Product Line \mysource{\seiwhitepaperspl\mypage{5}}}
			\mycitebegin A \emph{software product line} is
			\begin{itemize}
				\item a set of software-intensive systems \uncover<2->{
					\begin{exampletight}{}
						aka.\ products or variants
					\end{exampletight}
				}
				\item that share a common, managed set of features \uncover<3->{
					\begin{exampletight}{}
						common set, but not all products have all features in common
					\end{exampletight}
				}
				\item satisfying the specific needs of a particular market segment or mission \uncover<4->{
					\begin{exampletight}{}
						aka.\ domain 
					\end{exampletight}
				}
				\item and that are developed from a common set of core assets in a prescribed way.\myciteend \uncover<5->{
					\begin{exampletight}{}
						aka.\ planned, structured reuse 
					\end{exampletight}
				}
				\mysource{\href{https://resources.sei.cmu.edu/library/asset-view.cfm?assetID=513819}{Software Engineering Institute, Carnegie Mellon University}}
			\end{itemize}
		\end{definition}
	\end{fancycolumns}
\end{frame}
% variants / platforms / domain / variant generation
% configurable software, highly-configurable software, variable software, software variation

\subsection{Product-Line Engineering}
\begin{frame}[label=SoftwareProductLineEngineering]{\myframetitle}
	\begin{fancycolumns}[animation=none]
		\begin{definition}{Product-Line Engineering\mysource{\sple\mypage{14}}}
			\mycite{\emph{Software product-line engineering} is a paradigm to develop software applications (software-intensive systems and software products) using software platforms and mass customization.}
%			\begin{itemize}
%			\item aka.\ product-line development \mysource{\seiwhitepaperspl\mypage{207}}
%			\end{itemize}
		\end{definition}
		\uncover<2->{\begin{note}{Promises of Product Lines \mysource{\fospl\mypages{9--10}}}
			\begin{itemize}
				\item tailor-made
				\item reduced costs 
				\item improved quality 
				\item reduced time-to-market 
			\end{itemize}
		\end{note}}
	\nextcolumn
		\uncover<3->{\begin{notetight}{Idea of Product-Line Engineering}
			Reduce effort per product by means of an up-front investment for the product line:
			
			\pic[width=\linewidth,trim=30 55 30 50,clip]{product-line-vs-single-system}
		\end{notetight}}
		\uncover<4->{\begin{definition}{Single-System Engineering}
			\begin{itemize}
			%\item aka.\ single-system development \mysource{\seiwhitepaperspl\mypage{7}}
			\item[] classical software development that is not considered as product-line engineering
			\end{itemize}
		\end{definition}}
	\end{fancycolumns}
\end{frame}

% product-line hall of fame
% success stories: Boing, Bosch, Hewlett Packard, Toshiba, General Motors \mysource{\fospl}

% TODO \subsection{Running Examples?}
% \fospl: data management for embedded systems, product line of a graph library

% not software, not running: financial services (or in \lecturemodeling?), bikes, shoes, notebooks, ...
% Linux, Graph Product Line \lectureruntime\ or \lecturecloneandown, Configurable Database, Elevator Product Line, Printer product lines, Car configurators, software ecosystem (Browser plug-ins, IDE plug-ins, https://www.sciencedirect.com/science/article/pii/S0950584914001220)

% elevators: pictures from various elevators (OVGU, UULM, Bern), invisible end-user-visible features (e.g., double click for deselection)
% if you do this at home: do not select all stops before you know it is working

% how many examples in first lecture? move certain examples in later lectures? if so, which ones?
%\subsection{Automotive Systems}
% car configurators
% history? number of variants over time?
%\subsection{Notebooks}
% lenovo, microsoft, apple
%\subsection{Printer (Firmware)}
% real printers, 3d printers
% 30 printers per year, more examples
% XKCD: all-in-one paper processor
%\subsection{Operating Systems}
% windows, linux!!!, android
% Linux: where is it used? on which principal hardware? how many instances are running world wide? how many commits and developers? how many versions have been release and since when? screenshot how it looks like when linux is configured
% apps?
%\subsection{Integrated Development Environments}
% eclipse
%\subsection{Browsers}
% plug-ins

% apps! in market store android, iOS
% ecos, packages in debian

%\subsection{Beyond Software}
% financial products by KfW, bikes, shoes, muesli, Subway, headphones, lego, detergents, furniture (handcrafted vs standardized vs product line), kitchens
% brompton: picture in ulm slides

% zoo of animals/tools

% historical development? exchangable parts, production lines, automated product lines, ...
% all-in-one solution vs custom development? software for German fire departments
% all-in-one application software vs embedded software
% reasons for custom development: 

% goal of the lecture

%\subsection{Features}

%\subsection{Software Product Lines}
%% product-line engineering
%\begin{frame}{Who Produces Only One Product?}
%	\href{https://pxhere.com/en/photo/920906}{\pic[width=.6\linewidth]{car-tower}}
%\end{frame}

%\subsection{Single System}
%% single-system engineering
%
%\begin{frame}{Greenfield Development? \deutschertitel{Auf der grünen Wiese?}}
%	\href{https://github.com/SoftVarE-Group/SlideTemplate/blob/main/pics/nature/may21-ulm.jpg}{\pic[width=.6\linewidth]{may21-ulm}}
%\end{frame}



% add illustration for variants/versions (space/time) by icons of word/excel/powerpoint/one note/... over the years




% Apel 2013, Page 49
%Compile-time variability is decided before or at compile time.
%Load-time variability is decided after compilation when the program is started.
%With run-time variability, decisions can be made and changed during program
%execution.



% explain configurators vs selectors? for example, iPhone is available is produced in any combination and then selled, still there are selectors to find the right product easily. cf. ebay/amazon




% history on product lines? how old are product lines and ideas discussed in this lecture?





\lessonslearned{
	\item mass customization\\= mass production + customization
	\item features, products, domains
	\item software product lines
	\item product-line engineering
}{
	\item \fospl\mychapter{1}\mypages{3--15} --- introduction close to this lecture, further examples
	\item \seiwhitepaperspl\mypages{2--8} --- what is not a product line?
	% TODO add further reference? \sple\mypage{14}
}{
	\begin{itemize}
		\item What other examples of product lines do you know?
		\item Exemplify the differences between feature, product, domain, and product line for these examples.
		\item Are these product lines related to software?
	\end{itemize}
}

\section{Challenges of Product Lines}
\input{content/01b-challenges}
\lessonslearned{
	\item why are product lines challenging?
	\item selected challenges:
		\begin{enumerate}
		\item software clones
		\item feature traceability
		\item automated generation
		\item combinatorial explosion
		\item feature interactions
		\item continuous growth
		\end{enumerate}
}{
	\item[] see later lectures
}{
	\begin{itemize}
		\item Form groups of 2--3 students
		\item Explain 2--3 of the six challenges to your colleagues
		\item Can you find own examples for these challenges?
	\end{itemize}
}

\section{Course Organization}
\subsection{What You Should Know}

\begin{frame}{\myframetitle{}}
	\begin{fancycolumns}
		\begin{note}{Fundamentals of Software Engineering}
			\begin{itemize}
				\item development processes
				\item object-oriented programming
				\item design patterns
				\item UML class diagrams
				\item modularity
				\item web programming
			\end{itemize}
			\ifuniversity{magdeburg}{$\Rightarrow$ \emph{Software Engineering}}
		\end{note}
		\nextcolumn
		\begin{note}{Fundamentals of Theoretical Computer Science}
			\begin{itemize}
				\item set theory
				\item propositional logic
				\item complexity theory
			\end{itemize}
			\ifuniversity{magdeburg}{
				$\Rightarrow$ \emph{Logik}\\
				$\Rightarrow$ \emph{Grundlagen der Theoretischen Informatik I}
			}
		\end{note}
		\begin{note}{Exercise}
			solid programming skills in Java
			
			\ifuniversity{magdeburg}{
				$\Rightarrow$ \emph{Einführung in die Informatik}\\
				$\Rightarrow$ \emph{Algorithmen und Datenstrukturen}
			}
		\end{note}
	\end{fancycolumns}
\end{frame}

\subsection{What You Will Learn}

\begin{frame}{\myframetitle{}}
	\footnotesize	\lectureseriesoverview[1]
\end{frame}

\subsection{What You Might Need}

\begin{frame}{\myframetitle{}}
	\myframeicon{\mytitlesource{\fospl, \featureide}}
	\begin{fancycolumns}
		\begin{exampletight}{Recommended Literature for Lecture \& Exercise}
			\centering
			\parbox{0.49\linewidth}{
				\centering
				\pic[width=\linewidth]{cover-fospl}
				\emph{theory-focused}
			}
			\parbox{0.475\linewidth}{
				\centering
				\pic[width=\linewidth]{cover-featureide}
				\emph{practice-oriented}
			}
		\end{exampletight}
		\nextcolumn
		\begin{exampletight}{Recommended Tool Support for the Exercise}
			\centering
			\picDark[width=\linewidth]{featureide-feature-model-editor}\\[.5ex]
			\pic[width=0.25\linewidth]{featureide-logo}
		\end{exampletight}
	\end{fancycolumns}
\end{frame}

\subsection{Credit for the Slides}

\begin{frame}{\myframetitle{}}
	\ifuniversity{anonymous}{\mynote{}{\centering\huge Anonymous Authors}}
	\unlessuniversity{anonymous}{
		\myframeicon{\href{https://github.com/SoftVarE-Group/Course-on-Software-Product-Lines}{\pic[scale=.75]{cc-by-sa}}}
	}
	\begin{fancycolumns}[columns=3,animation=none]
		\unlessuniversity{anonymous}{
			\begin{note}{Thomas Thüm}
				\centering
				\href{https://www.uni-ulm.de/en/in/sp/team/thuem/}{\adjincludegraphics[height=.45\textheight,trim={.125\width} 0 {.125\width} 0,clip]{thomas-thuem}}
				
				\small Professor at TU Braunschweig
				
				software engineering
				
				FeatureIDE team leader
			\end{note}
		}
		\nextcolumn
		\unlessuniversity{anonymous}{
			\begin{note}{Timo Kehrer}
				\centering
				\href{https://seg.inf.unibe.ch/people/timo/}{\pic[height=.45\textheight]{timo-kehrer}}
				
				\small Professor at University of Bern
				
				software engineering
				
				~
			\end{note}
		}
		\nextcolumn
		\unlessuniversity{anonymous}{
			\begin{note}{Elias Kuiter}
				\centering
				\href{https://www.dbse.ovgu.de/en/Staff/Elias+Kuiter.html}{\pic[height=.45\textheight]{elias-kuiter}}
				
				\small PhD student in Magdeburg
				
				feature-model analysis
				
				FeatureIDE core developer
			\end{note}
		}
	\end{fancycolumns}
\end{frame}
%\inputuniversity{content/01c-course}
\lessonslearned{
	\item focus: how to implement features
	\item focus: how to model valid combinations
	\item focus: how to do quality assurance
	\item course organization
}{
	\item \fospl\ --- best book for this lecture
	\item \featureide\ --- more practical guide on tool support
}{
	\begin{itemize}
		\item Ask questions on the course organization!
		\unlessuniversity{magdeburg}{\item Form teams for the practical tasks}
	\end{itemize}
}

\faq{
	\item What is a software product line, feature, product/variant, domain?
	\item What is customization, handcrafting, mass production, mass customization?
	\item What are example for each of those?
	\item What is the one-size-fits-all or swiss-army-knife principle? \deutsch{Eierlegende Wollmilchsau}
	\item What is the difference between product-line engineering and single-system engineering?
	\item What are advantages of product-line engineering?
}{
	\item What are software clones,\\feature traceability,\\automated generation,\\ combinatorial explosion,\\feature interactions,\\and continuous growth?
	\item Why are those challenging when developing product lines?
	\item What are examples for these six fundamental challenges?
	\item How do those challenges interact with each other?
	\item How complex is the Linux kernel?
	\item At which pace is the Linux kernel developed?
}{
	\color{gray}\item What you should know?
	\item What you will learn?
	\item What you might need?
	\item Who are the authors of this course?
	\item How is this course organized?
}

\mode<beamer>{
	\addtocounter{framenumber}{-1}
	\begin{frame}{\inserttitle}
		\lectureseriesoverview[\insertlecturenumber]
	\end{frame}

	%\addtocounter{framenumber}{-1}
	%\againtitle % TODO does not work as we have redefined maketitle
}


% TODO L02 RUNTIME VARIABILITY AND DESIGN PATTERNS

\ifuniversity{recording}{\date{April 21, 2023}\setpicture[275]{apr21-o25b}\setcopyright{}}
\ifuniversity{ulm}{\date{April 20, 2023}\setpicture[275]{apr21-o25b}}
\ifuniversity{magdeburg}{\setpicture[300]{magdeburg-canal}\setcopyright{}}
\ifuniversity{bern}{\setpicture[50]{unibe_200512_00027_1200}}
\ifuniversity{paderborn}{\date{April 17, 2024}\setpicture[0]{image7}}
\ifuniversity{braunschweig}{\date{October 23, 2024}}

\author{Timo Kehrer, Thomas Thüm, Elias Kuiter}
\lecture{Runtime Variability and Design Patterns}{runtime}

\section{Configuration of Runtime Variability}
\input{content/02a-configuration}
\lessonslearned{
	\item External setting of configuration options through command-line parameters, preference dialogs, configuration files
	\item Validity of combinations may be affected by dependencies between features
}{
	\item \fospl\mychapter{3}\mypages{47--49}\\--- brief introduction of binding times
	% TODO better literature pointers?
}{
	\begin{itemize}
		\item Do you know any practical examples making use of runtime variability?
		\item How does the configuration take place?
		\item Is the configuration checked for validity?
	\end{itemize}
}

\section{Realization of Runtime Variability}
\input{content/02b-implementation}
\lessonslearned{
	\item Global (immutable) variables or (lengthy) parameter lists
	\item Reconfiguration at runtime is possible (in principle)
	\item Variability is spread over the entire program
	\item Variable parts are always delivered
}{
	\item \fospl\mychapter{4}
	\item \featureide\mysection{17.1}
	% TODO more details on what to expect in this literature
}{
	\begin{itemize} % die aufgaben sind ein bisschen theoretisch
		\item Why are code scattering, tangling and replication problematic?
		\item What are the problems of variable parts being always delivered?
	\end{itemize}
}

\section{Design Patterns for Variability}
\input{content/02c-patterns}
\lessonslearned{
	\item Variability through object-orientation and design patterns
	\item Extension through delegation vs.\ inheritance
	\item Limitations and drawbacks w.r.t.\ feature combinations
}{
	\item \oosoftwareconstruction\mychapter{3--4}
	\item \gof\mychapter{1--4}
	% TODO more details on what to expect in this literature
}{
	\begin{itemize}
		\item What characterizes a modular software design and why can it support variability?
		\item In which sense are object-oriented solutions more modular than simple conditional statements?
		\item Do you know of other design patterns supporting variability?
	\end{itemize}
}

\faq{
	\item What is variability, runtime variability, and binding time?
	\item What is the connection between variability, variability-intensive systems, and software product lines?
	\item How can configuration options be supplied?
	\item How can command-line parameters, preference dialogs, and configuration files be used to offer variability?
	\item What is the principle behind configuration options and when does the configuration happen?
	\item What are potential features of a graph library?
	\item What are examples for runtime variability? When are configuration options specified? By whom?
	\item When is the validity of configuration options checked?
}{
	\item How to realize runtime variability in source code?
	\item What is the best strategy?
	\item How can (immutable) global variables and method parameters be used to realize variability?
	%\item How to trace features in non-variable implementations?
	%\item How can configuration options be realized (for a graph library)?
	\item Why is the development of runtime variability challenging?
	\item What are code scattering, code tangling, and code replication?
	\item Does runtime variability allow reconfiguration at runtime?
	\item How to develop new features or variants with runtime variability? Exemplify!
	\item What are (dis)advantages of runtime variability?
	\item When (not) to use runtime variability?
}{
	\item What are design patterns (used for)?
	\item Why are design patterns relevant for variable software?
	\item Which design patterns are relevant for variable software?
	\item What are intent, motivation, and idea for template method, abstract factory, and decorator pattern?
	\item Given an example class diagram, which pattern is applied or should be applied? Why?
	\item What is the diamond problem? How does it affect variable software?
	\item Is multiple inheritance useful to combine features? Exemplify!
	\item What are limitations of design patterns (for variable software)?
}

\mode<beamer>{
	\addtocounter{framenumber}{-1}
	\begin{frame}{\inserttitle}
		\lectureseriesoverview[\insertlecturenumber]
	\end{frame}

	%\addtocounter{framenumber}{-1}
	%\againtitle % TODO does not work as we have redefined maketitle
}


% TODO L03 COMPILE-TIME VARIABILITY WITH CLONE-AND-OWN

\ifuniversity{recording}{\date{April 22, 2023}\setpicture[200]{apr21-south}\setcopyright{}}
\ifuniversity{ulm}{\date{April 27, 2023}\setpicture[200]{apr21-south}}
\ifuniversity{magdeburg}{\setpicture[70]{ovgu-autumn2}\setcopyright{Photo: Hannah Theile (OVGU)}}
\ifuniversity{bern}{\setpicture[50]{unibe_00434_200610_1200}}
\ifuniversity{paderborn}{\date{April 24, 2024}\setpicture[325]{image8}}
\ifuniversity{braunschweig}{\date{October 30, 2024}}
\ifuniversity{indonesia}{\date{February 3, 2025}}

\author{Timo Kehrer, Thomas Thüm, Elias Kuiter}
\lecture{Compile-Time Variability with Clone-and-Own}{cloneandown}

\section{Compile-Time Variability and Clone-and-Own}
\subsection{Problems with Runtime Variability}

\begin{frame}<3>{Recap: How to Implement Software Product Lines?}
	\frameImplementSPLs
\end{frame}

\begin{frame}<1>{Recap: Variability and Binding Times}
	\frameVariabilityAndBindingTimes
\end{frame}

\begin{frame}<1-2>[label=PrinciplesAndProblemsOfRuntimeVariability]
	\frametitle<1-2>{Recap: Runtime Variability}
	\frametitle<3>{\myframetitle}
	\begin{fancycolumns}[t]
		\begin{note}{Basic Principles}
			{\bf Variability with Configuration Options:}
			\begin{itemize}
				\item Conditional statements controlled by configuration options
				\item Global variables vs.\ method parameters
			\end{itemize}
			{\bf Object-Orientation and Design Patterns:}
			\begin{itemize}
				\item Template Method
				\item Abstract Factory
				\item Decorator
			\end{itemize}	
		\end{note}
	\nextcolumn
		\frameRuntimeVariabilityProblems
	\end{fancycolumns}	
\end{frame}

\subsection{Compile-Time Variability}

\begin{frame}{\myframetitle}
	\begin{fancycolumns}[widths={50},animation=none]
		\frameRuntimeVariabilityProblems		
	\nextcolumn
		\uncover<2->{\begin{definition}{Compile-Time Variability\mysource{\fospl\mypage{49}}}
			\mycite{Compile-time variability is decided before or at compile time.}
		\end{definition}}
		\uncover<3->{\begin{note}{}
			{\bf Goals:}
			\begin{itemize}
				\item Only required source code is compiled
				\item Smaller and highly optimized variants
			\end{itemize}				
			{\bf Challenge:}
			\begin{itemize}
				\item How to implement options and alternatives (i.e.,~variability)?
			\end{itemize}	
		\end{note}}
		\uncover<4->{\begin{note}{In this Lecture}
			Simple concepts and techniques for a few variants
		\end{note}}
	\end{fancycolumns}	
\end{frame}

\subsection{Ad-Hoc Clone-and-Own}

\begin{frame}{Ad-Hoc Clone-and-Own}
	\begin{fancycolumns}[widths={47},animation=none]
		\begin{note}{Clone-and-Own\mysource{\seiwhitepaperspl\mypage{7}}}
			\mycite{Suppose you are developing a new system that seems very similar to one you have built before. You borrow what you can from your previous effort, modify it as necessary, add whatever it takes, and field the product, which then assumes its own maintenance trajectory separate from that of the first product. What you have done is what is called \emph{clone and own}. You certainly have taken economic advantage of previous work; you have reused a part of another system. But now you have two entirely different systems, not two systems built from the same base. This is again \emph{ad hoc reuse}.}
		\end{note}
	\nextcolumn
		\begin{example}{Cloning Whole Products}
			~\hfill\pic[width=.2\linewidth]{130}\hfill\pic[width=.2\linewidth]{230}\hfill~
		\end{example}
		\pause
		\begin{definition}{Clone-and-Own} % TODO source missing
			\begin{itemize}
				\item New variants of a software system are created by copying and adapting an existing variant
				\item Afterwards, cloned variants evolve independently of each other
			\end{itemize}	
		\end{definition}
	\end{fancycolumns}	
\end{frame}

\subsection{Variability with Clone-and-Own}

\begin{frame}[fragile]{Example for Ad-Hoc Clone-and-Own}
	\myframeicon{\pic[scale=0.25,page=6]{graphs}}%
	\begin{fancycolumns}[b,columns=3,widths={43,32},animation=none]
\begin{codetight}{}
class Graph {
	List nodes = new ArrayList();
	List edges = new ArrayList();

	Edge add(Node n, Node m) {
		Edge e = new Edge(n, m);
		e.weight = new Weight();
		nodes.add(n); nodes.add(m); edges.add(e);
		return e;
	}
	Edge add(Node n, Node m, Weight w) {
		Edge e = new Edge(n, m);
		e.weight = w;
		nodes.add(n); nodes.add(m); edges.add(e);
		return e;
	}
	void print() {
		for (int i = 0; i < edges.size(); i++) {
			((Edge) edges.get(i)).print();
		}
	}
}
\end{codetight}
	\nextcolumn
		\begin{example}{}
			initial graph implementation providing weighted graphs
		\end{example}
\begin{codetight}{}
class Edge {
	Node a, b;
	Weight weight = new Weight();

	Edge(Node a, Node b) {
		this.a = a; this.b = b;
	}
	void print() {
		a.print(); b.print();
		weight.print();
	}
}
\end{codetight}
\begin{codetight}{}
class Weight {
	void print() {...}
}
\end{codetight}
	\nextcolumn
\begin{codetight}{}
public class Node {
	int id = 0;

	void print() {
		System.out.print(id);
	}
}
\end{codetight}
	\end{fancycolumns}
\end{frame}

\begin{frame}[fragile]{Alice's Clone: Unweighted Graphs}
	\myframeicon{\picDark[scale=0.25]{alice} \pic[scale=0.25,page=2]{graphs}}%
	\begin{fancycolumns}[b,columns=3,widths={43,32},animation=none]
\begin{codetight}{}
class Graph {
	List nodes = new ArrayList();
	List edges = new ArrayList();

	Edge add(Node n, Node m) {
		Edge e = new Edge(n, m);
		@|e.weight = new Weight();|@
		nodes.add(n); nodes.add(m); edges.add(e);
		return e;
	}
	@|Edge add(Node n, Node m, Weight w) {
		Edge e = new Edge(n, m);
		e.weight = w;
		nodes.add(n); nodes.add(m); edges.add(e);
		return e;
	}|@
	void print() {
		for (int i = 0; i < edges.size(); i++) {
			((Edge) edges.get(i)).print();
		}
	}
}
\end{codetight}
	\nextcolumn
		\begin{example}{}
			Alice works with unweighted graphs: she copies and adapts the basic implementation
		\end{example}
\begin{codetight}{}
class Edge {
	Node a, b;
	@|Weight weight = new Weight();|@

	Edge(Node a, Node b) {
		this.a = a; this.b = b;
	}
	void print() {
		a.print(); b.print();
		@|weight.print();|@
	}
}
\end{codetight}
\begin{codetight}{}
@|class Weight {
	void print() {...}
}|@
\end{codetight}
	\nextcolumn
\begin{codetight}{}
public class Node {
	int id = 0;

	void print() {
		System.out.print(id);
	}
}
\end{codetight}
	\end{fancycolumns}
\end{frame}

\begin{frame}[fragile]{Alice's Clone: Unweighted Graphs}
	\myframeicon{\picDark[scale=0.25]{alice} \pic[scale=0.25,page=2]{graphs}}%
	\begin{fancycolumns}[b,columns=3,widths={43,32},animation=none]
\begin{codetight}{}
class Graph {
	List nodes = new ArrayList();
	List edges = new ArrayList();

	Edge add(Node n, Node m) {
		Edge e = new Edge(n, m);
		nodes.add(n); nodes.add(m); edges.add(e);
		return e;
	}
	void print() {
		for (int i = 0; i < edges.size(); i++) {
			((Edge) edges.get(i)).print();
		}
	}
}
\end{codetight}
	\nextcolumn
		\begin{example}{}
			Alice works with unweighted graphs: she copies and adapts the basic implementation
		\end{example}
\begin{codetight}{}
class Edge {
	Node a, b;

	Edge(Node a, Node b) {
		this.a = a; this.b = b;
	}
	void print() {
		a.print(); b.print();
	}
}
\end{codetight}
	\nextcolumn
\begin{codetight}{}
public class Node {
	int id = 0;

	void print() {
		System.out.print(id);
	}
}
\end{codetight}
	\end{fancycolumns}
\end{frame}

\begin{frame}[fragile]{Bob's Clone: Colored Graphs}
	\myframeicon{\picDark[scale=0.25]{bob} \pic[scale=0.25,page=12]{graphs}}%
	\begin{fancycolumns}[b,columns=3,widths={43,26},animation=none]
\begin{codetight}{}
public class Graph {
	List nodes = new ArrayList();
	List edges = new ArrayList();

	Edge add(Node n, Node m) {
		Edge e = new Edge(n, m);
		nodes.add(n); nodes.add(m); edges.add(e);
		return e;
	}
	void print() {
		for (int i = 0; i < edges.size(); i++) {
			((Edge) edges.get(i)).print();
		}
	}
}
\end{codetight}
	\nextcolumn
		\begin{example}{}
			Bob works with colored graphs: he is a colleague of Alice and knows her variant, so he copies and adapts Alice's variant
		\end{example}
\begin{codetight}{}
class Edge {
	Node a, b;

	Edge(Node a, Node b) {
		this.a = a; this.b = b;
	}
	void print() {
		a.print(); b.print();
	}
}
\end{codetight}
	\nextcolumn
\begin{codetight}{}
public class Node {
	int id = 0;
	~Color color = new Color();~

	void print() {
		~Color.setDisplayColor(color);~
		System.out.print(id);
	}
}
\end{codetight}
\begin{codetight}{}
~public class Color {
	static void setDisplayColor(Color c) {...}
}~
\end{codetight}
	\end{fancycolumns}
\end{frame}

\subsection{Problems of Clone-and-Own}

\begin{frame}{Why is Clone-and-Own Problematic?}
	~\hfill
	\pic[scale=0.25,page=6]{graphs}
	\hfill
	\picDark[scale=0.25]{alice}
	\pic[scale=0.25,page=2]{graphs}
	\hfill
	\picDark[scale=0.25]{bob}
	\pic[scale=0.25,page=12]{graphs}
	\hfill~
\end{frame}

\begin{frame}{Clone-and-Own Problems: Feature Combinations}
	\myframeicon{\picDark[scale=0.25]{eve} \pic[scale=0.25,page=16]{graphs}}%
	\vspace{5mm}

	\begin{fancycolumns}[widths={42}]
		\begin{example}{}
			Eve has a new requirement:
			she wants to work with graphs which are both colored and weighted
		\end{example}
	\nextcolumn
		\begin{note}{}
			\begin{itemize}
				\item Where to start from?
				\item Does Eve know about Bob's and Alice's variants?
				\item If so, how to avoid repeating the work that has been already done by Alice and Bob, respectively?
			\end{itemize}
		\end{note}
	\end{fancycolumns}

	~

	\uncover<2->{
		~\hfill
		\pic[scale=0.25,page=6]{graphs}
		\hfill
		\picDark[scale=0.25]{alice}
		\pic[scale=0.25,page=2]{graphs}
		\hfill
		\picDark[scale=0.25]{bob}
		\pic[scale=0.25,page=12]{graphs}
		\hfill~
	}
\end{frame}

\begin{frame}[fragile]{Clone-and-Own Problems: Evolution \& Maintenance}
	\myframeicon{\pic[scale=0.2,page=6]{graphs}\\[-9mm]\pic[scale=0.2,page=26]{graphs}~~~}%
	\begin{fancycolumns}[b,widths={43}]
		\begin{example}{}
			Maintainers of the initial variant refactor the code of the basic implementation
		\end{example}
\begin{codetight}{}
public class Graph {
	...
	Edge add(Node n, Node m) {
		Edge e = new Edge(n, m);
		nodes.add(n); nodes.add(m); edges.add(e);
		e.weight = new Weight();
		return e;
	}
	Edge add(Node n, Node m, Weight w) {
		?|Edge e = new Edge(n, m);|?
		?|nodes.add(n); nodes.add(m); edges.add(e);|?
		@Edge e = add(n, m);@
		e.weight = w;
		return e;
	}
	...
}
\end{codetight} % TODO order of statements not consistent with prior examples: shall we revert the changed order in prior examples?
	\nextcolumn
		\begin{note}{}
			\begin{itemize}
				\item Who informs Alice, Bob and Eve about the improvement?
				\item How do they know whether the improvement is relevant for them?
				\item If so, how to propagate the improvement to their variant?
			\end{itemize}
		\end{note}

		~

		\picDark[scale=0.15]{alice}
		\pic[scale=0.15,page=2]{graphs}
		\hfill
		\picDark[scale=0.15]{bob}
		\pic[scale=0.15,page=12]{graphs}
		\hfill
		\picDark[scale=0.15]{eve}
		\pic[scale=0.15,page=16]{graphs}
	\end{fancycolumns}
\end{frame}

\subsection{Software Clones}

\begin{frame}<2>{Recap: \myframetitle\ \mytitlesource{\lectureintroduction}}
	\frameSoftwareClones
\end{frame}

\begin{frame}[b]{Types of \myframetitle}
	\begin{fancycolumns}[b,height=\textheightwithtitle]
		\begin{definition}{Types of Software Clones\mysource{\softwareclonedetection\mypage{1167}}}
			\begin{itemize}
			\item Type 1: identical except whitespaces and comments
			\item Type 2: syntactically similar (e.g., changed identifiers, \ldots)
			\item Type 3: copied with modifications (e.g., inserted or removed statements)
			\item Type 4: similar functionality without textual similarities
			\end{itemize}
		\end{definition}
		\begin{example}{Cloning Parts of Software}
			~\hfill\pic[width=.2\linewidth]{pants-blue}\hfill\pic[width=.2\linewidth]{pants-red}\hfill~
		\end{example}
	\nextcolumn
		\begin{note}{Relevant Types for Clone-and-\emph{Own}?}
			\begin{itemize}
			\item Type 1: may happen if clones diverge and comments need to reflect actual changes
			\item Type 2: may happen if clones diverge and identifier names are not appropriate anymore
			\item Type 3: actually necessary for clone-and-own
			\item Type 4: may happen if same functionality is implemented again (simply unknown or merge/cherrypick infeasible) \mysource{see \lecturemodeling}
			\end{itemize}
			Every difference is an obstacle for future maintenance (cf.\ merge and cherrypick)
		\end{note}
		\begin{example}{Cloning Whole Products (Clone-and-Own)}
			~\hfill\pic[width=.2\linewidth]{130}\hfill\pic[width=.2\linewidth]{230}\hfill~
		\end{example}
	\end{fancycolumns}
\end{frame}

\subsection{Discussion}

\begin{frame}{Discussion of Clone-and-Own}
	\pic[scale=0.2,page=6]{graphs}
	\hfill
	\picDark[scale=0.2]{alice}
	\pic[scale=0.2,page=2]{graphs}
	\hfill
	\picDark[scale=0.2]{bob}
	\pic[scale=0.2,page=12]{graphs}
	\hfill
	\picDark[scale=0.2]{eve}
	\pic[scale=0.2,page=16]{graphs}
	\hfill
	\pic[scale=0.2,page=26]{graphs}

	\begin{fancycolumns}[columns=2,widths={45,55}]
		\begin{note}{Advantages}
			\begin{itemize}
				\item Simple and straightforward approach
				\item Rapid exploration of new ideas
				\item No upfront investments
			\end{itemize}
		\end{note}
	\nextcolumn
		\begin{note}{Disadvantages}
			\begin{itemize}
				\item No structured and systematic reuse (copy \& edit)
				\item No flexible combination of features
				\item Maintenance quickly becomes impractical
			\end{itemize}
		\end{note}
	\end{fancycolumns}	
	\begin{fancycolumns}[columns=3,widths={15,70}]
	\nextcolumn
		\uncover<3->{\begin{note}{Towards Managed Clone-and-Own}
			\begin{itemize}
				\item How can we better manage such clone-and-own development?
				\item The traditional answer: Software Configuration Management
				\item In the sequel: Software Configuration Management in practice
			\end{itemize}
		\end{note}}
	\nextcolumn
	\end{fancycolumns}
\end{frame}

% TODO add \pic[width=\linewidth,page=24]{lego}

\lessonslearned{
	\item Compile-time variability is decided before or at compile time
	\item In clone-and-own, new variants of a software system are created by copying and adapting an existing variant
	\item Simple paradigm, but suffering from maintenance problems in the long run
}{
	\item\fospl\mysection{3.1.1}\mypages{48--49}\\--- brief introduction of compile-time variability
	\item\softwareclonedetection\mysection{2}\mypages{1167--1168}\\--- brief introduction to software clones, their types, reasons, (dis)advantages
	\item\virtualplatform\ --- brief introduction to ad-hoc clone-and-own (L0)
	% TODO better literature pointers on clone-and-own?
}{
	\begin{itemize}
		\item What are the reasons why clone-and-own is very popular in practice?
		\item What is the order of magnitude of the number of variants that can be reasonably maintained in clone-and-own?
		\item Have you ever applied the principle of clone-and-own? If so, where and how?
	\end{itemize}
}

\section{Clone-and-Own with Version Control}
\input{content/03b-versioncontrol}
\lessonslearned{
	\item Software configuration management as a traditional discipline of managing the evolution of variability-intensive systems
	\item Version control systems as a widespread tool supporting clone-and-own in practice
}{
	\item \fospl\mysection{5.1}\mypages{99--105}\\--- introduction of variability with version control (not~explicitly calling it clone-and-own)
	\item \staplesandhill\ --- experience report on managed clone-and-own with version control and build systems
	\item\virtualplatform\ --- brief introduction to clone-and-own with version control (L1)
	% TODO add literature? \summerville\mychapter{25}
}{
	\begin{itemize}
		\item Which software configuration management concepts are supported by version control systems? % TODO rather abstract question
		\item Do you know other version control systems than Git?  % TODO my feeling was that most students do not know anything beyond git and it does hot help to discuss this in groups
		\item If so, in which way are they different from Git?
		% TODO what about asking about differences between forks and branches? and what would be different with forks? when do they prefer forks/branches?
	\end{itemize}
}

\section{Clone-and-Own with Build Systems}
\input{content/03c-buildsystems}
\lessonslearned{
	\item Variability through build scripts
	\item Granularity of clones: Individual files
	\item Combination of files\\$\neq$ combination of features
}{
	\item \fospl\mysection{5.2.2 and 5.2.4}\mypages{106--110}\\--- brief introduction of clone-and-own with build systems (not~explicitly calling it clone-and-own)
	\item \staplesandhill\ --- experience report on managed clone-and-own with version control and build systems
}{
	\begin{itemize}
		\item Which software configuration management concepts are supported by build systems?
		\item What are the commonalities and differences of clone-and-own with version control and clone-and-own with build systems?
		\item What are the strengths and weaknesses?
		% TODO \item What is the effort to convert ad-hoc clone-and-own into managed clone-and-own?
	\end{itemize}
}

\faq{
	\item What are problems of runtime variability?
	\item What is compile-time variability?
	\item What is (ad-hoc) clone-and-own?
	\item How to develop new features or variants? Exemplify!
	\item What are (dis)advantages of (ad-hoc) clone-and-own?
	\item When (not) to use (ad-hoc) clone-and-own?
	\item What is better runtime or compile-time variability?
}{
	\item What is software configuration management and version control (used~for)?
	\item What is the difference between version, revision, variant, configuration, baseline, release, merge, cherrypick, revision control?
	\item How can version control be used for clone-and-own? Illustrate!
	\item How to develop new features or variants? Exemplify!
	\item What are (dis)advantages of version control for variability?
	\item When (not) to use clone-and-own with version control?
	\item Shall we use version control at all?
}{
	\item What are build systems (used~for)?
	\item How can build systems be used for clone-and-own? Illustrate!
	\item How to develop new features or variants? Exemplify!
	\item What are (dis)advantages of build systems for variability?
	\item When (not) to use clone-and-own with build systems?
	\item Shall we use build systems at all?
	\item What is the granularity of clones for all three techniques?
	\item What is the effort to migrate from ad-hoc clone-and-own to managed clone-and-own?
}

\mode<beamer>{
	\addtocounter{framenumber}{-1}
	\begin{frame}{\inserttitle}
		\lectureseriesoverview[\insertlecturenumber]
	\end{frame}

	%\addtocounter{framenumber}{-1}
	%\againtitle % TODO does not work as we have redefined maketitle
}

