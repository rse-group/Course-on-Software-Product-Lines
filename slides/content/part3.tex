% TODO L09 FEATURE INTERACTIONS

\ifuniversity{recording}{\date{June 21, 2023}\setpicture[350]{may21-west2}\setcopyright{}}
\ifuniversity{ulm}{\date{June 22, 2023}\setpicture[350]{may21-west2}}
\ifuniversity{magdeburg}{\setpicture[35]{ovgu-winter3}\setcopyright{Photo: Hannah Theile (OVGU)}}
\ifuniversity{bern}{\setpicture[1]{unibe_00571_200610_1200}}
\ifuniversity{paderborn}{\date{June 12, 2024}\setpicture[0]{pressimage3}}
\ifuniversity{braunschweig}{\date{December 11, 2024}}

\author{Thomas Thüm, Timo Kehrer, Elias Kuiter}
\lecture{Feature Interactions}{interactions}

% TODO move to Lecture 10?
%\subsection{Recap: Software Quality}
%\begin{frame}{\myframetitle} % caution: slide copied from testing lecture
%	\rightorleft{
%		\mydefinition{Quality \mysource{\ludewiglichter}}{Quality is the entirety of properties and characteristics of a product or process that indicate adequacy with respect to given requirements.}
%		\mydefinition{Quality Assurance \mysource{\ludewiglichter}}{Quality assurance \deutsch{Qualitätssicherung} are all activities with the goal to improve the quality.}
%	}{
%		\vspace{-12mm}
%		\centering
%		\pic[width=\linewidth,trim=0 240 0 300,clip]{andy-hunt}
%		\vspace{-7mm}
%		
%		\mynote{Andy Hunt \mysource{\thepragmaticprogrammer}}{\mycite{No one in the brief history of computing has ever written a piece of perfect software. It's unlikely that you'll be the first.}}
%		% co-authored The Pragmatic Programmer, known for the Agile Manifesto
%	}
%\end{frame}

\section{What is a Feature Interaction?}
\subsection{Examples for Feature Interactions}

% TODO more ideas for examples
% beamer+notebook
% esp+abs: both control brakes and motor

\begin{frame}{An Interaction when Customizing Clothes}
	\begin{fancycolumns}[widths={33},animation=none]
		\begin{example}{Customization of Clothes}
			\begin{itemize}
			\item platforms to create and buy clothes
			\item creator preselects clothes with certain colors
			\item customization with pictures in different colors
			\item where is the problem?
			\end{itemize}
		\end{example}
	\nextcolumn
		\mywhite{uulm Shop \mysource{\href{https://uulm.myspreadshop.de/collections}{myspreadshop.de}}}{
			\pic[height=40mm]{spreadshirt-shop-2}
			\hfill
			\pic[height=40mm]{spreadshirt-shop-3}
		} % TODO exampletight with white background in darkmode
	\end{fancycolumns}
\end{frame}
\begin{frame}{An Interaction when Customizing Clothes}
	\begin{fancycolumns}
		\begin{exampletight}{The Problem: Unwanted Interaction of Colors}
			~

			\centering\picDark[height=50mm]{spreadshirt-email-1}

			~
		\end{exampletight}
	\nextcolumn
		\begin{exampletight}{The Solution: Choose Other Colors}
			~

			\centering\picDark[height=50mm]{spreadshirt-email-4}

			~
		\end{exampletight}
	\end{fancycolumns}
\end{frame}
\begin{frame}{An Interaction when Customizing Clothes}
	\begin{fancycolumns}
		\mywhite{The Problem: Unwanted Interaction of Colors}{
			\centering\pic[height=45mm,trim=50 3550 690 3340,clip]{spreadshirt-order}
		} % TODO exampletight with white background in darkmode
	\nextcolumn
		\mywhite{The Solution: Choose Other Colors}{
			\centering\pic[height=45mm,trim=50 3255 690 3635,clip]{spreadshirt-order}
		} % TODO exampletight with white background in darkmode
	\end{fancycolumns}
	\uncover<3->{\begin{note}{}
		\centering seems that contrast is checked for each order (cf.\ application engineering)\\and not for each published design (cf.\ domain engineering)
	\end{note}}
\end{frame}
% TODO add links to keynote?

\begin{frame}{An Interaction of Android Apps}
	\begin{fancycolumns}[widths={67},animation=none]
		\mywhite{}{\only<1-2|handout:0>{\pic[width=\linewidth,page=11,trim=40 30 280 100,clip]{2021/2021-09-08-SPLC-Keynote}}%
		\only<3->{\pic[width=\linewidth,page=11,trim=40 30 40 100,clip]{2021/2021-09-08-SPLC-Keynote}}}%
		\only<4->{\begin{note}{}%
			\centering which of those 3.5 million Android apps interact?\\where to document?\\whom to blame?%
		\end{note}}%
	\nextcolumn
		\begin{example}{Skype vs BabyMonitor}
			\begin{itemize}
			\item Skype app installed and used for years
			\item BabyMonitor installed, carefully tried and used for months
			\item BabyMonitor can call any other number (i.e., works without internet)
			\item automatic update of the Skype app
			\item update causes a question to be asked for every call
			\item \alt<-2>{what is the problem?}{found baby crying as no one answered the dialog}
			\end{itemize}
		\end{example}
	\end{fancycolumns}
\end{frame}
% TODO use pictures directly, add picture showing all apps
% TODO add links to keynote?

%\begin{frame}{Can We Trust Our Scans?}
%	\centering\pic[width=\linewidth,page=28,trim=40 30 40 70,clip]{2021/2021-09-08-SPLC-Keynote}
%\end{frame}
% TODO add scanner example? is it really dependent on multiple options? or only on one particular setting

\subsection{Feature Interactions}
\begin{frame}{\myframetitle}
	\begin{fancycolumns}
		\begin{definition}{Feature Interaction\mysource{\fospl\mypage{214}}}
			\mycite{A \emph{feature interaction} between two or more features is an
emergent behavior that cannot be easily deduced from the behaviors associated
with the individual features involved.}

			for short: interaction
		\end{definition}
		\begin{definition}{Inadvertent Interaction\mysource{\fospl\mypage{214}}}
			\mycite{An \emph{inadvertent feature interaction} occurs when a feature influences the behavior of another feature in an unexpected way (for example, regarding the expected control flow, program or data state, or visible behavior).}

			here simplified as: wanted vs unwanted
		\end{definition}
	\nextcolumn
		\begin{definition}{Feature-Interaction Problem\mysource{\fospl\mypage{214}}}
			\mycite{The \emph{feature-interaction problem} is to detect, manage, and resolve (inadvertent) feature interactions among features.}
		\end{definition}
		\begin{note}{Feature Interactions}
			\begin{itemize}
				\item detection discussed in next two lectures \mysource{\lectureanalyses\ and \lecturetesting}
				\item resolving interactions (see Part II)
				\item managing interactions (see Part III)
			\end{itemize}
		\end{note}
		\begin{note}{What's Next?}
			\begin{itemize}
				\item interactions due to the absence of features
				\item interactions in source code
				\item interactions with the base code
			\end{itemize}
		\end{note}
	\end{fancycolumns}
\end{frame}

\begin{frame}{A Common Interaction of Toasters}
	\begin{fancycolumns}[animation=none]
		\only<1|handout:0>{\pic[width=\linewidth]{toast1}}%
		\only<2|handout:0>{\pic[width=\linewidth]{toast2}}%
		\only<3-|handout:1>{\pic[width=\linewidth]{toast3}}%
		\uncover<5->{\begin{example}{}
			\centering no interaction for two toasts (i.e., \emph{$T_1 \pand T_2$} shown) and for no toasts (i.e., $\pnot T_1 \pand \pnot T_2$ not shown)
		\end{example}}
	\nextcolumn
		\only<4->{\pic[width=\linewidth]{toast4}}%
		\uncover<6->{\begin{example}{}
			\centering unwanted interaction for one toast\\(i.e., \emph{$T_1 \pand \pnot T_2$} shown and  $\pnot T_1 \pand T_2$ not shown)
		\end{example}}
	\end{fancycolumns}
\end{frame}

\subsection{Example Interactions with Preprocessors}
\begin{frame}{\myframetitle}
	\begin{fancycolumns}[widths={70}]
		\only<1|handout:0>{\pic[width=\linewidth,page=1,trim=20 20 20 40,clip]{preprocessor-wilderness}}%
		\only<2->{\pic[width=\linewidth,page=2,trim=20 20 20 40,clip]{preprocessor-wilderness}}%
	\nextcolumn
		\begin{example}{No Interaction?}\setlength\leftmargini{3mm}
			\begin{itemize}
				\item configuration for undirected, weighted edges
				\item product can be compiled
				\item what is the problem?
			\end{itemize}
		\end{example}
	\end{fancycolumns}
\end{frame}
\begin{frame}{\myframetitle}
	\begin{fancycolumns}[widths={70}]
		\pic[width=\linewidth,page=4,trim=20 20 20 40,clip]{preprocessor-wilderness}
	\nextcolumn
		\begin{example}{Static Interaction}\setlength\leftmargini{3mm}
			\begin{itemize}
				\item configuration for undirected, unweighted edges
				\item product cannot be compiled due to static feature interaction
				\item field \emph{weight} used for undirected edges but defined in feature \emph{Weighted}
				\item occurs whenever features \emph{Directed} and \emph{Weighted} are both not selected
			\end{itemize}
		\end{example}
	\end{fancycolumns}
\end{frame}
\begin{frame}{\myframetitle}
	\begin{fancycolumns}[widths={70}]
		\pic[width=\linewidth,page=3,trim=20 20 20 40,clip]{preprocessor-wilderness}
	\nextcolumn
		\begin{example}{Other Static Interaction}\setlength\leftmargini{3mm}
			\begin{itemize}
				\item configuration for directed, weighted edges
				\item product cannot be compiled due to static feature interaction
				\item semicolon and bracket missing for every configuration with feature \emph{Directed}
				\item feature \emph{Directed} has inadvertent interaction with base code
			\end{itemize}
		\end{example}
	\end{fancycolumns}
\end{frame}
\begin{frame}{\myframetitle}\setlength\leftmargini{3mm}
	\begin{fancycolumns}[widths={70}]
		\pic[width=\linewidth,page=5,trim=20 20 20 40,clip]{preprocessor-wilderness}
	\nextcolumn
		\begin{example}{Dynamic Interaction?}
			\begin{itemize}
				\item again: configuration for undirected, weighted edges
				\item product can be compiled but test fails
				\item not a static interaction
				\item defect in the base code
				\item no interaction at all
			\end{itemize}
		\end{example}
		\begin{note}{Detection of Interactions}
			\begin{itemize}
				\item static interactions\\\mysource{\lectureanalyses}
				\item dynamic interactions\\\mysource{\lecturetesting}
				\item next: interactions of more than two features
			\end{itemize}
		\end{note}
	\end{fancycolumns}
\end{frame}

%\subsection{Unwanted and Wanted Interactions} % Desired + Undesired
% \href{https://github.com/SoftVarE-Group/Slides/blob/main/2021/2021-09-08-SPLC-Keynote.pdf}{\mycite{Every unwanted feature interaction waits to be fixed or at least documented in form of a constraint.}} T:SPLC21 

%\subsection{Pairwise Interactions}
\subsection{Higher-Order Interactions}
\begin{frame}{\myframetitle}
	\begin{fancycolumns}
		\begin{definition}{Kinds of Interactions}
			\begin{itemize}
				\item wanted and unwanted interactions
				\item static and dynamic interactions
				\item one-wise interactions (interaction of features with the base code and faulty features)
				\item pair-wise interactions (between two features)
				\item higher-order interactions (between more than two features)
			\end{itemize}
		\end{definition}
		\uncover<2->{\begin{example}{Variability Bug Database\mysource{\VBDb}}
			\begin{itemize}
				\item database of known feature interactions
				\item operating system Linux (43 interactions)
				\item web server Apache (23)
				\item system tool Busybox (18)
				\item 3D printer firmware Marlin (14)
			\end{itemize}
		\end{example}}
	\nextcolumn
		\begin{exampletight}{Patterns of Feature Interactions\mysource{\VBDb}}
			\pic[width=\linewidth]{variabilitybugdatabase-patterns}
		\end{exampletight}
	\end{fancycolumns}
\end{frame}
% TODO example from the The Variability Bug Database?

\begin{frame}{Interaction on Data and Control Flow \mytitlesource{\essentialconfigurationcomplexity}}\setlength\leftmargini{3mm}
	\begin{fancycolumns}[widths={70}]
		\mywhite{}{\essentialconfigurationcomplexitylink{\pic[width=\linewidth,page=2,trim=55 495 225 75,clip]{2016/2016-ASE-Meinicke}}}
	\nextcolumn
		\begin{note}{How do Features Interact?}
			\begin{itemize}
				\item given a program with runtime variability
				\item given one test case (i.e., concrete inputs)
				\item how much does the execution depend on the configuration?
				\item how many values for each variable? (green)
				\item how many different control flows? (red)
				\item blue color not relevant here (minimal overhead during simultaneous execution)
			\end{itemize}
		\end{note}
	\end{fancycolumns}	
\end{frame}
\begin{frame}[label=ExecutionTracesInConfigurableSystems]{Execution Traces in Configurable Systems \mytitlesource{\essentialconfigurationcomplexity}}
	\mywhite{}{\essentialconfigurationcomplexitylink{\pic[width=.95\linewidth,page=8,trim=55 520 55 55,clip]{2016/2016-ASE-Meinicke}}}

	\begin{note}{}
		\centering insights: not all features interact. some statements may lead to higher interactions than others.
	\end{note}
\end{frame}

% TODO explain duality between partial configurations and conjunctions of literals (cf. elevator product line by Varshosaz et al.)

\lessonslearned{
	\item feature interaction and feature-interaction problem
	\item wanted/unwanted, static/dynamic, one-wise/pair-wise/higher-order
	\item examples: customization of clothes, Android apps, toaster, preprocessor code, runtime variability, Variability Bug Database
}{
	\item \fospl\mychapter{9}\mypages{213--217}
}{
	Do you know further examples of feature interactions?
}

\section{How to Handle Feature Interactions?}
\input{content/09b-handlinginteractions}
\lessonslearned{
	\item Adaptation of feature model to avoid (undesired) feature interactions. 
	\item Strategies to implement coordination code for known feature interactions.
	\item Discussion of the strengths and weaknesses of each of the strategies.
}{
	\item Kästner et al.: On the impact of the optional feature problem: Analysis and case studies. SPLC 2009.
	\item \fospl\mychapter{9} % TODO add pages
}{
	Looking back at our graph library and the feature interaction between $ShortestPath$ and $Weighted$. Which strategy would you choose to handle this interaction? Why?
	
	Can you think of other feature interactions for the graph library (you may also add additional features)? Again, how would you handle them? 
}

\section{How to Avoid Feature Interactions?}
\input{content/09c-avoidinginteractions}
\lessonslearned{
	\item reduction of variability
	\item which features are actually needed?
	\item documentation of interactions that cannot be avoided
}{
	\item[] \fospl % TODO add chapter and pages
}{
	Who checks the compatibility of Lenovo products?
}

\faq{
	\item What are (inadvertent) feature interactions? Give examples!
	\item Are feature interactions limited to product lines?
	\item What is the feature-interaction problem? Why is it critical for product lines?
	\item What is the difference of static/dynamic, one-wise/pair-wise/higher-order interactions?
	\item What are typical patterns of interactions?
	\item How can features interact on control flow and data?
}{
	\item How to resolve feature interactions?
	\item What is the optional feature problem?
	\item What are typical goals when resolving feature interactions?
	\item Name/explain strategies to resolve feature interactions!
	\item What are (dis)advantages of those strategies?
}{
	\item How to cope with feature interactions?
	\item How to reduce variability?
	\item What are unused, unnecessary, and shopping-list-features?
	\item How to document feature interactions?
}

\mode<beamer>{
	\addtocounter{framenumber}{-1}
	\begin{frame}{\inserttitle}
		\lectureseriesoverview[\insertlecturenumber]
	\end{frame}

	%\addtocounter{framenumber}{-1}
	%\againtitle % TODO does not work as we have redefined maketitle
}


% TODO L10 PRODUCT-LINE ANALYSES

\ifuniversity{recording}{\date{June 9, 2023}\setpicture[70]{ovgu-spring}\setcopyright{Photo: Jana Dünnhaupt (OVGU)}}
\ifuniversity{ulm}{\date{June 29, 2023}\setpicture[225]{may21-west4}}
\ifuniversity{magdeburg}{\setpicture[70]{ovgu-winter4}\setcopyright{Photo: Jana Dünnhaupt (OVGU)}}
\ifuniversity{bern}{\setpicture[25]{unibe_00135_201310_1200}}
\ifuniversity{paderborn}{\date{June 19, 2024}\setpicture[175]{pressimage5}}
\ifuniversity{braunschweig}{\date{December 18, 2024}}

\author{Elias Kuiter, Thomas Thüm, Timo Kehrer}
\lecture{Product-Line Analyses}{analyses}

\section{Analysis Strategies}
\input{content/10a-strategies}
\lessonslearned{
	\item product-line analyses are needed for quality assurance
	\item \emph{product-based}: simple, but does not scale
	\item \emph{feature-based}: fairly simple, but misses interactions
	\item \emph{family-based}: efficient, but most complex
}{
	\item \fospl\mychapter{10}
	\item \analysisstrategies
}{
	Can you imagine other analysis strategies than product-based, feature-based, and family-based?
	How could such strategies look like?
}

\section{Analyzing Feature Mappings}

\newcommand{\notleftright}{\mathrel{\ooalign{$\Leftrightarrow$\cr\hidewidth$/$\hidewidth}}}

\subsection{Automated Analysis of Feature Mappings}

\begin{frame}{\myframetitle}
	\begin{fancycolumns}[widths={45,55}]
		\begin{note}{Recap: A Typical Product Line}
			\begin{itemize}
				\item embedded or systems programming (e.g., Linux)
				\item implemented with conditional compilation
				\begin{itemize}
					\item build systems (e.g., KBuild)
					\item preprocessors (e.g., CPP)
				\end{itemize}
				\item feature traceability only implicit\\
					$\Rightarrow$ there is code scattering and tangling
			\end{itemize}
		\end{note}
		\begin{definition}{Recap: Feature Mapping}
			\begin{itemize}
				\item specifies which features correspond to which artifacts (individual files/lines, components/feature modules/aspects)
				\item connects the problem space to the solution space
			\end{itemize}
		\end{definition}
		\nextcolumn
		\begin{example}{Asking Questions About the Feature Mapping}
			\begin{itemize}
				\item Is the code even included in any product?
				\item Are there contradictory or unnecessary preprocessor annotations in the code?
				\item How scattered and tangled is the code?
				\item \ldots
			\end{itemize}
		\end{example}
	\end{fancycolumns}
\end{frame}

\begin{frame}{\myframetitle}
	\begin{fancycolumns}[widths={45,55}]
		\begin{exampletight}{Running Example: Graph Product Line}
			\centering
			\featureDiagram{Graph,concrete[Node,concrete,mandatory[Colored,concrete,optional]][Edge,concrete,mandatory[Directed,concrete,optional][Undirected,concrete,optional][Hyper,concrete,optional]]}
			$\pnot (Directed \pand Undirected)$\\
			$Hyper \pimplies Undirected$\\
			$Directed \por Hyper$
		\end{exampletight}
	\nextcolumn
		\myexample{An Undirected Hypergraph}{
			\centering
			\picDark[width=.8\linewidth]{hypergraph}
		}
	\end{fancycolumns}
\end{frame}

\subsection{Presence Conditions}

\begin{frame}[b,fragile]{\myframetitle}
	\vspace*{-3ex}
	\begin{fancycolumns}[t,columns=3,widths={40,23,37},animation=none]
		\begin{definition}{Presence Condition}
			A \emph{presence condition (PC)} for a code location (i.e., a line or file) is a formula that describes the circumstances under which the code location is included in a product.
		\end{definition}
		\begin{note}{}
			\begin{itemize}
				\item useful for implementation techniques with code scattering and tangling
				\item e.g., build systems (file PCs) or preprocessors (line PCs)
				\item here: line PCs for the C preprocessor
			\end{itemize}
		\end{note}
	\nextcolumn
		\uncover<3->{
			\begin{exampletight}{Presence Conditions}
				\small\vspace*{0.5ex}
				\begin{flushright}

					\uncover<4->{{$\top$}\\
					$\top$\\
					$Colored$\\
					$Colored$\\
					$Colored$\\
					$\top$\\}
					\uncover<5->{$\top$\\
					$\top$\\
					$Directed$\\
					$Directed$\\
					$\pnot Dir \pand Hyper$\\}
					\uncover<6->{$\pnot Dir \pand Hy \pand Un$\\
					$\pnot Dir \pand Hy \pand Un$\\
					$\pnot Dir \pand Hy \pand \pnot Un \pand Dir$\\
					$\pnot Dir \pand Hy \pand \pnot Un \pand Dir$\\
					$\pnot Dir \pand Hy \pand \pnot Un \pand Dir$\\}
					\uncover<7->{$\pnot Dir \pand \pnot Hy$\\
					$\pnot Dir \pand \pnot Hy \pand \pnot Dir$\\
					$\pnot Dir \pand \pnot Hy \pand \pnot Dir$\\
					$\pnot Dir \pand \pnot Hy \pand \pnot Dir$\\
					$\pnot Dir \pand \pnot Hy$\\
					$\top$}
				\end{flushright}
			\end{exampletight}
		}
	\nextcolumn
		\begin{uncoverenv}<2->
			\begin{cpptight}[basicstyle=\small]{\texttt{graph.cpp}}
class Node {
	string label;
#ifdef COLORED
	string color;
#endif
};

class Edge {
#ifdef DIRECTED
	Node fromNode, toNode;
#elifdef HYPER
#ifdef UNDIRECTED
	set<Node> nodeSet;
#elifdef DIRECTED
	map<Node, set<Node>> nodeMap;
#endif
#else
#ifndef DIRECTED
	pair<Node, Node> nodePair;
#endif
#endif
};
			\end{cpptight}
		\end{uncoverenv}
	\end{fancycolumns}
\end{frame}

\subsection{Detecting Dead Code}

\begin{frame}[b,fragile]{\myframetitle}
	\vspace*{-4ex}
	\begin{fancycolumns}[t,columns=3,widths={40,23,37},animation=none]
		\begin{definition}{Dead Code}
			A line or file of code is \emph{dead} when

			\begin{itemize}
				\item no product includes it.
				\item or, equivalently:\\
					its presence condition $PC$ is contradictory (i.e., $PC \mimplies \bot$).
			\end{itemize}
		\end{definition}
		\begin{note}{}
			calculated by querying a \emph{satisfiability solver} whether $PC$ is not satisfiable (i.e., $\pnot SAT(PC)$)
		\end{note}
		\uncover<3->{
			\begin{note}{What causes dead code?}
				\begin{itemize}
					\item confusion due to nested \texttt{\#ifdef}
					\item domain modeling mistakes
					\item can be intended! \mysource{\href{https://dl.acm.org/doi/10.1145/3442391.3442406}{Hentze~et~al.~2021}}
				\end{itemize}
			\end{note}
		}
	\nextcolumn
		\uncover<2->{
			\begin{exampletight}{Presence Conditions}
				\small\vspace*{0.5ex}
				\begin{flushright}
					{\color{gray}$\top$}\\
					{\color{gray}$\top$}\\
					{\color{gray}$Colored$}\\
					{\color{gray}$Colored$}\\
					{\color{gray}$Colored$}\\
					{\color{gray}$\top$}\\
					{\color{gray}$\top$}\\
					{\color{gray}$\top$}\\
					{\color{lecturegreen}$Directed$}\\
					{\color{gray}$Directed$}\\
					{\color{lecturegreen}$\pnot Dir \pand Hyper$}\\
					{\color{gray}$\pnot Dir \pand Hy \pand Un$}\\
					{\color{gray}$\pnot Dir \pand Hy \pand Un$}\\
					{\color{lecturered}$\pnot Dir \pand Hy \pand \pnot Un \pand Dir$}\\
					{\color{lecturered}$\pnot Dir \pand Hy \pand \pnot Un \pand Dir$}\\
					{\color{lecturered}$\pnot Dir \pand Hy \pand \pnot Un \pand Dir$}\\
					{\color{gray}$\pnot Dir \pand \pnot Hy$}\\
					{\color{gray}$\pnot Dir \pand \pnot Hy \pand \pnot Dir$}\\
					{\color{gray}$\pnot Dir \pand \pnot Hy \pand \pnot Dir$}\\
					{\color{gray}$\pnot Dir \pand \pnot Hy \pand \pnot Dir$}\\
					{\color{gray}$\pnot Dir \pand \pnot Hy$}\\
					{\color{gray}$\top$}
				\end{flushright}
			\end{exampletight}
		}
	\nextcolumn
		\begin{uncoverenv}<2->
			\begin{cpptight}[basicstyle=\small]{\texttt{graph.cpp}}
class Node {
	string label;
#ifdef COLORED
	string color;
#endif
};

class Edge {
#ifdef DIRECTED
	Node fromNode, toNode;
#elifdef HYPER
#ifdef UNDIRECTED
	set<Node> nodeSet;
#elifdef DIRECTED
	@map<Node, set<Node>> nodeMap;@
#endif
#else
#ifndef DIRECTED
	pair<Node, Node> nodePair;
#endif
#endif
};
			\end{cpptight}
		\end{uncoverenv}
	\end{fancycolumns}
\end{frame}

\subsection{Detecting Superfluous Annotations}

\begin{frame}[b,fragile]{\myframetitle}
	\vspace*{-4ex}
	\begin{fancycolumns}[t,columns=3,widths={40,23,37},animation=none]
		\begin{definition}{Superfluous Annotation}
			An annotation is \emph{superfluous}

			\begin{itemize}
				\item when it can be omitted without consequences.
				\item or, equivalently:\\
					its presence condition $PC$ is implied by the enclosing presence condition $PC'$ (i.e., $PC' \mimplies PC$).
			\end{itemize}
		\end{definition}
		\begin{note}{}
			calculated by querying a \emph{satisfiability solver} whether $PC' \pand \pnot PC$ is not satisfiable (i.e., $\pnot SAT(PC' \pand \pnot PC)$)
		\end{note}
		\uncover<2->{
			\begin{example}{}
				\begin{itemize}
					\item {\color{lecturegreen}$PC' = \pnot Dir \pand \pnot Hy$}
					\item {\color{lecturered}$PC = \pnot Dir \pand \pnot Hy \pand \pnot Dir$}
				\end{itemize}
			\end{example}
		}
	\nextcolumn
	\uncover<2->{
		\begin{exampletight}{Presence Conditions}
			\small\vspace*{0.5ex}
			\begin{flushright}
				{\color{gray}$\top$}\\
				{\color{gray}$\top$}\\
				{\color{gray}$Colored$}\\
				{\color{gray}$Colored$}\\
				{\color{gray}$Colored$}\\
				{\color{gray}$\top$}\\
				{\color{gray}$\top$}\\
				{\color{gray}$\top$}\\
				{\color{gray}$Directed$}\\
				{\color{gray}$Directed$}\\
				{\color{gray}$\pnot Dir \pand Hyper$}\\
				{\color{gray}$\pnot Dir \pand Hy \pand Un$}\\
				{\color{gray}$\pnot Dir \pand Hy \pand Un$}\\
				{\color{gray}$\pnot Dir \pand Hy \pand \pnot Un \pand Dir$}\\
				{\color{gray}$\pnot Dir \pand Hy \pand \pnot Un \pand Dir$}\\
				{\color{gray}$\pnot Dir \pand Hy \pand \pnot Un \pand Dir$}\\
				{\color{lecturegreen}$\pnot Dir \pand \pnot Hy$}\\
				{\color{lecturered}$\pnot Dir \pand \pnot Hy \pand \pnot Dir$}\\
				{\color{lecturered}$\pnot Dir \pand \pnot Hy \pand \pnot Dir$}\\
				{\color{lecturered}$\pnot Dir \pand \pnot Hy \pand \pnot Dir$}\\
				{\color{lecturegreen}$\pnot Dir \pand \pnot Hy$}\\
				{\color{gray}$\top$}
			\end{flushright}
		\end{exampletight}
	}
	\nextcolumn
		\begin{uncoverenv}<2->
			\begin{cpptight}[basicstyle=\small]{\texttt{graph.cpp}}
class Node {
	string label;
#ifdef COLORED
	string color;
#endif
};

class Edge {
#ifdef DIRECTED
	Node fromNode, toNode;
#elifdef HYPER
#ifdef UNDIRECTED
	set<Node> nodeSet;
#elifdef DIRECTED
	map<Node, set<Node>> nodeMap;
#endif
#else
#ifndef @DIRECTED@
	pair<Node, Node> nodePair;
#endif
#endif
};
			\end{cpptight}
		\end{uncoverenv}
	\end{fancycolumns}
\end{frame}

\subsection{Joining the Problem and Solution Space}

\begin{frame}[fragile]{\myframetitle}
	\begin{fancycolumns}[widths={67,33},animation=none]
		\begin{note}{}
			\begin{itemize}
				\item right now, we only consider \emph{line PCs} (from the preprocessor)
				\item but: a line is only included if its file is included, too\\
					$\Rightarrow$ we also have to consider \emph{file PCs} (from the build system)
				\item also: we want to ignore invalid configurations\\
					$\Rightarrow$ we also have to consider the \emph{feature model} $FM$
				\item idea: \emph{join} feature model, file, and line presence condition:\\
					$PC_{location} \defeq \Phi(FM) \pand PC_{file} \pand PC_{line}$
			\end{itemize}
		\end{note}
		\uncover<2->{
			\begin{exampletight}{Suppose we have the feature model \ldots}
				\centering
				\featureDiagram{Graph,concrete[Node,concrete,mandatory[Colored,concrete,optional]][Edge,concrete,mandatory[Directed,concrete,optional][Undirected,concrete,optional][Hyper,concrete,optional]]}
				$\pnot (Directed \pand Undirected) \pand (Hyper \pimplies Undirected) \pand (Directed \por Hyper)$
			\end{exampletight}
		}
	\nextcolumn
		\begin{uncoverenv}<3->
			\begin{cpptight}[basicstyle=\tiny]{\ldots\ and two files: \texttt{node.cpp} \ldots}
class Node {
	string label;
#ifdef COLORED
	string color;
#endif
};
			\end{cpptight}
			\begin{cpptight}[basicstyle=\tiny]{\ldots\ and \texttt{edge.cpp}}
class Edge {
#ifdef DIRECTED
	Node fromNode, toNode;
#elifdef HYPER
#ifdef UNDIRECTED
	set<Node> nodeSet;
#elifdef DIRECTED
	map<Node, set<Node>> nodeMap;
#endif
#else
#ifndef DIRECTED
	pair<Node, Node> nodePair;
#endif
#endif
};
			\end{cpptight}
		\end{uncoverenv}
	\end{fancycolumns}
\end{frame}

\begin{frame}[b,fragile]{\myframetitle}
	\begin{fancycolumns}[t,columns=4,widths={31,20,20,29},animation=none]
		\emph{Problem Space}
		\uncover<5->{
		\begin{exampletight}{}
			\centering
			\featureDiagram{G,concrete[N,concrete,mandatory[C,concrete,optional]][E,concrete,mandatory[D,concrete,optional][U,concrete,optional][H,concrete,optional]]}
			$\pnot (Directed \pand Undirected)$\\
			$Hyper \pimplies Undirected$\\
			$Directed \por Hyper$
		\end{exampletight}
		\vspace*{-5ex}
		\begin{align*}
			\big\downarrow\Phi
		\end{align*}
		\vspace*{-5ex}
		\begin{exampletight}{}
			\vspace*{-3ex}
			\begin{align*}
				&Graph \pand Node \pand Edge \\
				\pand &\pnot (Directed \pand Undirected) \\
				\pand &(Hyper \pimplies Undirected)\\
				\pand &(Directed \por Hyper)
			\end{align*}
		\end{exampletight}
		}
	\nextcolumn
		\emph{Solution Space $\rightarrow$}
		\uncover<4->{
			\begin{exampletight}{File PC \tiny\texttt{node.cpp}}
				\Huge\centering\rotatebox{45}{$Node$}
				\vspace*{0.2ex}
			\end{exampletight}
			\begin{exampletight}{File PC \tiny\texttt{edge.cpp}}
				\vspace*{4.5ex}
				\Huge\centering\rotatebox{45}{$Edge$}
				\vspace*{4.5ex}
			\end{exampletight}
		}
	\nextcolumn
		\uncover<3->{
			\begin{exampletight}{Line PCs \tiny\texttt{node.cpp}}
				\tiny\vspace*{0.5ex}
				\begin{flushright}
					{$\top$}\\
					{$\top$}\\
					{$Colored$}\\
					{$Colored$}\\
					{$Colored$}\\
					{$\top$}
				\end{flushright}
			\end{exampletight}
			\begin{exampletight}{Line PCs \tiny\texttt{edge.cpp}}
				\tiny\vspace*{0.5ex}
				\begin{flushright}
					{$\top$}\\
					{$Directed$}\\
					{$Directed$}\\
					{$\pnot Dir \pand Hyper$}\\
					{$\pnot Dir \pand Hy \pand Un$}\\
					{$\pnot Dir \pand Hy \pand Un$}\\
					{$\pnot Dir \pand Hy \pand \pnot Un \pand Dir$}\\
					{$\pnot Dir \pand Hy \pand \pnot Un \pand Dir$}\\
					{$\pnot Dir \pand Hy \pand \pnot Un \pand Dir$}\\
					{$\pnot Dir \pand \pnot Hy$}\\
					{$\pnot Dir \pand \pnot Hy \pand \pnot Dir$}\\
					{$\pnot Dir \pand \pnot Hy \pand \pnot Dir$}\\
					{$\pnot Dir \pand \pnot Hy \pand \pnot Dir$}\\
					{$\pnot Dir \pand \pnot Hy$}\\
					{$\top$}
				\end{flushright}
			\end{exampletight}
		}
	\nextcolumn
		\begin{uncoverenv}<2->
			\begin{cpptight}[basicstyle=\tiny]{\texttt{node.cpp}}
class Node {
	string label;
#ifdef COLORED
	string color;
#endif
};
			\end{cpptight}
			\begin{cpptight}[basicstyle=\tiny]{\texttt{edge.cpp}}
class Edge {
#ifdef DIRECTED
	Node fromNode, toNode;
#elifdef HYPER
#ifdef UNDIRECTED
	set<Node> nodeSet;
#elifdef DIRECTED
	map<Node, set<Node>> nodeMap;
#endif
#else
#ifndef DIRECTED
	pair<Node, Node> nodePair;
#endif
#endif
};
			\end{cpptight}
		\end{uncoverenv}
	\end{fancycolumns}
\end{frame}

\begin{frame}[fragile]{\myframetitle}
	\begin{fancycolumns}[t,columns=4,widths={31,20,20,29},animation=none]
		\begin{exampletight}{Feature-Model Formula}
			\vspace*{-3ex}
			\color{lecturered}
			\begin{align*}
				&Graph \pand Node \pand Edge \\
				\pand &\pnot (Directed \pand Undirected) \\
				\pand &(Hyper \pimplies Undirected)\\
				\pand &(Directed \por Hyper)
			\end{align*}
		\end{exampletight}
	\nextcolumn
		\begin{exampletight}{File PC \tiny\texttt{edge.cpp}}
			\vspace*{4.5ex}
			\Huge\centering\rotatebox{45}{\color{lecturered}$Edge$}
			\vspace*{4.5ex}
		\end{exampletight}
	\nextcolumn
		\begin{exampletight}{Line PCs \tiny\texttt{edge.cpp}}
			\tiny\vspace*{0.5ex}
			\begin{flushright}
				{\color{gray}$\top$}\\
				{\color{gray}$Directed$}\\
				{\color{gray}$Directed$}\\
				{\color{gray}$\pnot Dir \pand Hyper$}\\
				{\color{gray}$\pnot Dir \pand Hy \pand Un$}\\
				{\color{gray}$\pnot Dir \pand Hy \pand Un$}\\
				{\color{gray}$\pnot Dir \pand Hy \pand \pnot Un \pand Dir$}\\
				{\color{gray}$\pnot Dir \pand Hy \pand \pnot Un \pand Dir$}\\
				{\color{gray}$\pnot Dir \pand Hy \pand \pnot Un \pand Dir$}\\
				{\color{gray}$\pnot Dir \pand \pnot Hy$}\\
				{\color{gray}$\pnot Dir \pand \pnot Hy \pand \pnot Dir$}\\
				{\color{lecturered}$\pnot Dir \pand \pnot Hy \pand \pnot Dir$}\\
				{\color{gray}$\pnot Dir \pand \pnot Hy \pand \pnot Dir$}\\
				{\color{gray}$\pnot Dir \pand \pnot Hy$}\\
				{\color{gray}$\top$}
			\end{flushright}
		\end{exampletight}
	\nextcolumn
		\begin{cpptight}[basicstyle=\tiny]{\texttt{edge.cpp}}
class Edge {
#ifdef DIRECTED
	Node fromNode, toNode;
#elifdef HYPER
#ifdef UNDIRECTED
	set<Node> nodeSet;
#elifdef DIRECTED
	map<Node, set<Node>> nodeMap;
#endif
#else
#ifndef DIRECTED
	@pair<Node, Node> nodePair;@
#endif
#endif
};
		\end{cpptight}
	\end{fancycolumns}
	\begin{alignat*}{3}
		PC_{location} \defeq~&\uncover<2->{\Phi(FM) &~\pand~& PC_{\texttt{edge.cpp}} &~\pand~& PC_{\texttt{pair<Node, Node> nodePair;}}\\}
		\uncover<3->{=~&G \pand N \pand E \pand \pnot (D \pand U) \pand (H \pimplies U) \pand (D \por H) &~\pand~& E &~\pand~& \pnot D \pand \pnot H \pand \pnot D\\}
		\uncover<4->{\mequals~&G \pand N \pand E \pand \pnot (D \pand U) \pand (H \pimplies U) \pand {\color{lecturered}(D \por H)} &~\pand~& E &~\pand~& {\color{lecturered}\pnot D \pand \pnot H} \pand \pnot D\\}
		\uncover<5->{\mimplies~&(D \por H) \pand \pnot D \pand \pnot H\\}
		\uncover<6->{\mimplies~&\bot \text{ -- so this code is dead after all!}}
	\end{alignat*}
\end{frame}

\subsection{Analyzing Feature Modules}

\begin{frame}[b]{\myframetitle}
	\vspace*{-3ex}
	\begin{fancycolumns}[widths={34,66},animation=none]
		\begin{exampletight}{}
			\featureDiagram{Store,abstract[Type,abstract,mandatory[SingleStore,concrete,alternative][MultiStore,concrete]][AccessControl,concrete,optional]}
		\end{exampletight}
		\uncover<2->{
			\begin{example}{Feature-Model Formula}
				\small
				$\Phi(FM) = Store \pand Type \pand (SS \por MS) \pand (\pnot SS \por \pnot MS)$
			\end{example}
			\begin{example}{Valid Configurations}
				\small
				\begin{fancycolumns}[animation=none]
					{$\{SS\}$}\\
					$\{SS, AC\}$
				\nextcolumn
					{$\{MS\}$}\\
					$\{MS, AC\}$\\
				\end{fancycolumns}
			\end{example}
		}
	\nextcolumn
		\uncover<3->{\picDark[width=\linewidth]{fop-csur-example}}
	\end{fancycolumns}
	\uncover<3->{
		\vspace*{-3ex}
		\begin{note}{}
			Is there dead code? Are there superfluous annotations?
		\end{note}
	}
\end{frame}

\begin{frame}[b]{\myframetitle}
	\vspace*{-3ex}
	\begin{fancycolumns}[widths={34,66},animation=none]
		\begin{exampletight}{}
			\featureDiagram{Store,abstract[Type,abstract,mandatory[SingleStore,concrete,alternative][MultiStore,concrete]][AccessControl,concrete,optional]}
		\end{exampletight}
		\begin{notetight}{Recap: 1:1 Feature Mapping!}
			\picDark[width=\linewidth,clip,trim=0 120 250 0]{feature_komposition1}
		\end{notetight}
	\nextcolumn
		\picDark[width=\linewidth]{fop-csur-example}
	\end{fancycolumns}
	\uncover<2->{
		\vspace*{-3ex}
		\begin{note}{}
			\sout{Is} \emph{Are} there \sout{dead code} \emph{dead features}? Are there \sout{superfluous annotations} \emph{redundant constraints}?
		\end{note}
	}
\end{frame}

\subsection{Feature-Mapping Analyses in FeatureIDE}

\begin{frame}{\myframetitle}
	\begin{fancycolumns}[widths={57,43}]
		\picDark[width=\linewidth]{featureide-dead-code}
		~
		\href{https://youtu.be/jVe7f32mLCQ?t=125}{demo video available} (minute 3 and 4): dead code block, superfluous annotations, generation of all products, error propagation, unit testing
	\nextcolumn
		\begin{note}{Discussion}
			\begin{itemize}
				\item we can now \emph{identify anomalies}:
				\begin{itemize}
					\item dead (unused) code
					\item mistakes in preprocessor annotations
					\item disagreements between problem and solution space
				\end{itemize}
				\item but: we only analyze the feature mapping and \emph{ignore the actual code}
				\begin{itemize}
					\item pro: simple, language-independent
					\item con: can only find simple anomalies
				\end{itemize}
				\item difficulty depends on the feature traceability (harder for conditional compilation than for FOP)
			\end{itemize}
		\end{note}
	\end{fancycolumns}
\end{frame}

\lessonslearned{
	\item feature-mapping analyses alleviate the impact of code scattering and tangling
	\item they are usually not necessary when there is good feature traceability
	\item they cannot detect bugs in the actual code
}{
	\item \fospl\mychapter{10}
}{
	Above, we assumed that we know all presence conditions already.
	How can we automatically extract presence conditions from code that uses the C preprocessor?
	What problems might occur?
	% - there might be #includes that contain macro definitions
	% - there might be macros to be expanded, which can get complex quickly
	% - it is not always trivial to distinguish feature macros from compiler- or system-specific macros
	% - there is #undef (or re-#define), which opens another can of worms
	% see works on TypeChef (C. Kästner), SuperC (P. Gazillo), FeatureCoPP (K. Ludwig), PCLocator (E. Kuiter)
}

\section{Analyzing Variable Code}
\subsection{Automated Analysis of Variable Code}

\begin{frame}{\myframetitle}
	\begin{fancycolumns}[t,widths={46}]
		\begin{example}{Asking Questions About\\the Feature Mapping \ldots}
			\begin{itemize}
				\item Are there contradictory or unnecessary preprocessor annotations in the code?
				\item Is the code even included in any product?
				\item If so, in how many products is the code included?
				\item \ldots
			\end{itemize}
		\end{example}
		\begin{note}{}
			only finds code-agnostic anomalies
		\end{note}
		\nextcolumn
		\begin{example}{\ldots\ and the Variable Code}
			\begin{itemize}
				\item Can every product be generated (e.g., compiled)?\\
					$\Rightarrow$ to find all \emph{syntax and type errors}
				\item Do all tests succeed for every product?\\
					$\Rightarrow$ to find some \emph{runtime and logic errors}
				\item Does every product adhere to its specification?\\
					$\Rightarrow$ to \emph{rule out} runtime and logic errors
				\item \ldots
			\end{itemize}
		\end{example}
		\begin{note}{}
			now: analyze (non-)functional properties of all products
		\end{note}
		\begin{example}{Today's Example}
			type checking for FOP and conditional compilation
		\end{example}
	\end{fancycolumns}
\end{frame}

\subsection{Variability-Aware Type Checking}

\subsubsection{Analyzing Feature Modules}

\begin{frame}{\myframetitle}
	\begin{fancycolumns}[widths={37,63},animation=none]
		\vspace*{-3ex}
		\begin{exampletight}{}
			\featureDiagram{Store,abstract[Type,abstract,mandatory[SingleStore,concrete,alternative][MultiStore,concrete]][AccessControl,concrete,optional]}
		\end{exampletight}
		\begin{example}{Feature-Model Formula}
			\small
			$\Phi(FM) = Store \pand Type \pand (SS \por MS) \pand (\pnot SS \por \pnot MS)$
		\end{example}
		\begin{example}{Valid Configurations}
			\small
			\begin{fancycolumns}[animation=none]
				{\color{black}$\{SS\}$}\\
				$\{SS, AC\}$
			\nextcolumn
				{\color{black}$\{MS\}$}\\
				$\{MS, AC\}$\\
			\end{fancycolumns}
		\end{example}
		\uncover<3->{
			\begin{note}{}
				Is there a type error in \emph{any} product?
				\uncover<4->{What about $\{SS, AC\}$?}
			\end{note}
		}
	\nextcolumn
		\only<2->{\picDark[width=\linewidth]{fop-csur-example-error-blind}}
	\end{fancycolumns}
\end{frame}
% TODO would be good to avoid the abbreviation SS here. what about only referring to concrete features here?

\begin{frame}[b]{\myframetitle}
	\begin{fancycolumns}[widths={37,63},animation=none]
		\vspace*{-3ex}
		\begin{definition}{Reachability Condition of $id$}
			guarantees that a given \emph{reference} to $id$ is also \emph{defined} somewhere:\\[1ex]
			$\Phi(FM) \mimplies (PC_{ref}^{id} \pimplies \bigvee_{def} PC_{def}^{id})$\\[1ex]
			or, with a SAT solver:\\
			\small
			$\pnot SAT(\Phi(FM) \pand PC_{ref} \pand \bigwedge_{def} \pnot PC_{def})$
		\end{definition}
		\uncover<2->{
			\begin{example}{}
				$\Phi(FM) \mimplies (AC \pimplies SS \por MS)$ holds,\\
				$\Phi(FM) \mimplies (AC \pimplies MS)$ does not\\[1ex]
				$\Rightarrow$ $\{SS, AC\}$ has no \texttt{readAll}!
			\end{example}
		}
		\uncover<3->{
			\begin{definition}{Type-Safe Product-Line}
				in a \emph{type-safe} SPL, all references must always be defined (i.e., \emph{all} reachability conditions must hold)\\
				{\color{gray}\ldots\ and many more conditions \ldots}
			\end{definition}
		}
	\nextcolumn
		\picDark[width=\linewidth]{fop-csur-example-error}
	\end{fancycolumns}
\end{frame}

\subsubsection{Analyzing Conditional Compilation}

\begin{frame}[fragile]{\myframetitle}
	\begin{fancycolumns}[columns=3,widths={37,29,34},animation=none]
		\begin{exampletight}{}
			\centering
			\featureDiagram{Graph,concrete[Directed,concrete,optional][Undirected,concrete,optional][Hyper,concrete,optional]}
			{\color{lectureblue}$\pnot (Directed \pand Undirected)$}\\
			{\color{lecturegreen}$Hyper \pimplies Undirected$}\\
			{\color{lecturered}$Directed \por Hyper$}
		\end{exampletight}
		\uncover<2->{
			\begin{definition}{Reachability Condition of $id$}
				$\Phi(FM) \mimplies (PC_{ref}^{id} \pimplies \bigvee_{def} PC_{def}^{id})$
			\end{definition}
		}
		\uncover<3->{
			\begin{definition}{Conflict Condition of $id$, def's $d_i$}
				guarantees that no definition of $id$ \emph{conflicts} with another:\\[1ex]
				$\Phi(FM) \mimplies \bigwedge_{d_1 \neq d_2} \pnot (PC_{d_1}^{id} \pand PC_{d_2}^{id}))$
			\end{definition}
		}
	\nextcolumn
	\uncover<2->{
		\begin{example}{Is $e.nodes$ reachable?}
			\small
			$\Phi(FM) \mimplies (\top \pimplies$\\
			$~~Dir \por (Hy \pand Un) \por (Hy \pand Dir))$\\[1ex]
			holds, because each graph is {\color{lecturered}directed} or an ({\color{lecturegreen}undirected}) {\color{lecturered}hypergraph}
		\end{example}
	}
		\uncover<3->{
			\begin{example}{Does $e.nodes$ conflict?}
				\small
				$\Phi(FM) \mimplies ($\\
				$~~\pnot (Dir \pand (Hy \pand Un))$\\
				$~\pand \pnot (Dir \pand (Hy \pand Dir))$\\
				$~~\pand \pnot ((Hy \pand Un) \pand (Hy \pand Dir)))$\\[1ex]
				holds, because a graph is {\color{lectureblue}never} {\color{lecturered}directed} and an ({\color{lecturegreen}undirected}) {\color{lecturered}hypergraph} {\color{lectureblue}at the same time}
			\end{example}
		}
		\uncover<4->{
			\begin{example}{}
				all reachable, no conflicts
			\end{example}
		}
	\nextcolumn
		\begin{cpptight}[basicstyle=\small]{\texttt{graph.cpp}}
class Node { ... };

class Edge {
#ifdef DIRECTED
	pair<Node, Node> nodes;
#endif
#ifdef HYPER
#ifdef UNDIRECTED
	set<Node> nodes;
#endif
#ifdef DIRECTED
	map<Node, set<Node>> nodes;
#endif
#endif
};

std::ostream& operator<<(
	std::ostream &s, const Edge &e) {
	return s << e.nodes;
}
		\end{cpptight}
	\end{fancycolumns}
\end{frame}

\subsubsection{Discussion}

\begin{frame}{\myframetitle}
	\begin{fancycolumns}
		\begin{note}{Just the Tip of the Iceberg}
			\begin{itemize}
				\item here, we only discussed reachability and conflict conditions
				\item but: actual type checking requires a table of all identifiers, their types, and their PCs (and a lot more SAT queries)
				\item the practical difficulty depends:
				\begin{itemize}
					\item FOP (due to superimposition)\\
						$\Rightarrow$ no conflict conditions required
					\item good feature traceability (e.g., FOP)\\
						$\Rightarrow$ trivial PCs, simpler implementation
					\item ignoring the feature model\\
						$\Rightarrow$ better performance (false positives!)
				\end{itemize}
			\end{itemize}
		\end{note}
	\nextcolumn
		\begin{example}{The TypeChef Project \mysource{\typechef}}
			\begin{itemize}
				\item a variability-aware lexer, parser framework, and type system for C code with \texttt{\#ifdef}'s
				\item skips preprocessing, instead builds an abstract syntax tree (AST) annotated with presence conditions
				\item \href{https://ckaestne.github.io/TypeChef/typechef-poster.png}{poster} with examples
				\item does it scale?\\[1ex]
				{\small Busybox (811 features): \mycite{We need 57 minutes to type check all modules.}} \mysource{\href{https://dl.acm.org/doi/10.1145/2384616.2384673}{ref}}\\[1ex]
				{\small Linux (6065 features): \mycite{We successfully parsed [it in] roughly 85 hours on a single machine.}} \mysource{\href{https://dl.acm.org/doi/10.1145/2048066.2048128}{ref}} % not practically possible, can be used to motivate testing in next lecture
			\end{itemize}
		\end{example}
	\end{fancycolumns}
\end{frame}

\subsection{Product-Line Analyses in the Wild}

\subsubsection*{Product-Line Complexity}

\begin{frame}{\myframetitle}
	\begin{fancycolumns}[animation=none]
		\begin{definition}{Six Classes of Product-Line Complexity \unlessuniversity{anonymous}{\mytitlesource{\href{https://youtu.be/qUuRp7_d0rU?t=1651}{Thüm~2021}}}}
			In a timeframe of 24h \ldots
			\begin{enumerate}
				\uncover<2->{\item[\color{gray}\textbf{NC}] {\color{gray}Products cannot be generated automatically}}
				\uncover<3->{\item[\textbf{C1}] All products can be generated and \emph{tested}}
				\uncover<4->{\item[\textbf{C2}] Not C1, but all \emph{products} can be \emph{generated}}
				\uncover<5->{\item[\textbf{C3}] Not C2, but all \emph{configurations} can be \emph{generated} (AllSAT)}
				\uncover<6->{\item[\textbf{C4}] Not C3, but the \emph{number of valid configurations} can be computed (\ssat{})}
				\uncover<7->{\item[\textbf{C5}] Not C4, but \emph{whether there is a valid configuration} can be computed (SAT)}
				\uncover<8->{\item[\textbf{C6}] It cannot be computed whether there is a valid configuration}
			\end{enumerate}
		\end{definition}
	\nextcolumn
		\begin{example}{Examples}
			% also put examples of analysis strategies here, and which kind of mapping/code analysis scales for which class?
			\begin{enumerate}
				\uncover<2->{\item[\color{gray}\textbf{NC}] {\color{gray}all product lines with mandatory custom development in application engineering\\(e.g., components and services with glue code, white-box frameworks)}}
				\uncover<3->{\item[\textbf{C1}] $< 2000$ products for 1 min per product}
				\uncover<4->{\item[\textbf{C2}] $< 90000$ products for 1 s per product}
				\uncover<5->{\item[\textbf{C3}] $< 10^{13}$ configurations for 1 ns per configuration}
				\uncover<6->{\item[\textbf{C4}] older versions of Linux/Automotive05}
				\uncover<7->{\item[\textbf{C5}] newer versions of Linux/Automotive05\\(see \evaluatingsharpsatsolvers)}
				\uncover<8->{\item[\textbf{C6}] No example known}
			\end{enumerate}
		\end{example}
	\end{fancycolumns}
\end{frame}

\subsubsection*{Automated Analysis \ldots}

\begin{frame}{\myframetitle}
	\begin{fancycolumns}[t,columns=3]
		\textbf{\lecturemodeling\partc}
		\begin{definition}{\ldots\ of Feature Models}
			analyze only the feature model
		\end{definition}
	\nextcolumn
		\textbf{\lectureanalyses\partb}
		\begin{definition}{\ldots\ of Feature Mappings}
			analyze the feature mapping (considering the feature model)
		\end{definition}
	\nextcolumn
		\textbf{\lectureanalyses\partc}
		\begin{definition}{\ldots\ of Variable Code}
			analyze the variable code (considering the feature model and feature mapping)
		\end{definition}
	\end{fancycolumns}
	\begin{fancycolumns}[t,columns=3]
		\begin{example}{}
			\begin{itemize}
				\item void, core/dead features
				\item decision propagation
				\item atomic sets, redundant constraints
				\item \ldots
			\end{itemize}
		\end{example}
	\nextcolumn
		\begin{example}{}
			\begin{itemize}
				\item dead code
				\item superfluous annotations
				\item degree of code scattering and tangling
				\item \ldots
			\end{itemize}
		\end{example}
	\nextcolumn
		\begin{example}{}
			\begin{itemize}
				\item parsing, type checking
				\item static analysis
				\item model checking, theorem proving
				\item \ldots
			\end{itemize}
		\end{example}
	\end{fancycolumns}
	\begin{fancycolumns}[t,columns=2,widths={32}]
	\nextcolumn
		\begin{note}{}
			here: \emph{family-based} analysis strategies for \emph{conditional compilation} and \emph{feature-oriented programming}
		\end{note}
	\end{fancycolumns}
\end{frame}
\lessonslearned{
	\item with family-based analyses of variable code, we can analyze (non-)functional properties of all products at once
	\item type checking all products at once is possible for product lines up to medium size
	\item for huge product lines (e.g., Linux), it is infeasible
}{
	\item \fospl\mychapter{10}
	\item \typechef
}{
	Suppose you have a preprocessor-based product line (with \texttt{\#ifdef}'s).
	If you could turn it into a single, large runtime-variable product (with \texttt{if}'s), you could use an off-the-shelf compiler to find any type error in any product.

	Is this possible? What problems might occur?
	\mysource{\sugarc}
}

\faq{
	\item How to find variability bugs?
	\item What is a program analysis? What are examples?
	\item What is a product-line analysis?
	\item What are principal strategies to analyze product lines? What are (dis-)advantages?
	\item Given a specific algorithm, classify its analysis strategy!
}{
	\item How to analyze feature mappings?
	\item What are potential problems in feature mappings?
	\item What are presence conditions, dead code, superfluous annotations?
	\item Shall we incorporate the feature model when analyzing feature mappings?
	\item Shall product-line analyses analyze problem and solution space separately?
	\item What is special when analyzing the feature mapping of feature modules?
	\item What are limitations of analyzing feature mappings?
	\item Given CPP source code, determine its presence conditions, dead code, and superfluous annotations!
}{
	\item What are (examples of) type errors?
	\item Why are type errors challenging to detect in product lines?
	\item What is a type-safe product line, reachability condition, conflict condition?
	\item How does the analysis complexity differ for real-world product lines?
	\item What are analyses for problem and solution space?
	\item Give examples for easy and difficult product lines in terms of analysis effort!
}

\mode<beamer>{
	\addtocounter{framenumber}{-1}
	\begin{frame}{\inserttitle}
		\lectureseriesoverview[\insertlecturenumber]
	\end{frame}

	%\addtocounter{framenumber}{-1}
	%\againtitle % TODO does not work as we have redefined maketitle
}


% TODO L11 PRODUCT-LINE TESTING

\ifuniversity{recording}{\date{June 21, 2023}\setpicture[200]{may23-south}\setcopyright{}}
\ifuniversity{ulm}{\date{July 6, 2023}\setpicture[200]{may23-south}}
\ifuniversity{magdeburg}{\setpicture[110]{ovgu-winter5}\setcopyright{Photo: Hannah Theile (OVGU)}}
\ifuniversity{bern}{\setpicture[1]{unibe_200305_00001_1200}}
\ifuniversity{paderborn}{\date{June 26, 2024}\setpicture[50]{pressimage7}}
\ifuniversity{braunschweig}{\date{January 8, 2025}}

\author{Thomas Thüm, Sebastian Krieter, Timo Kehrer, Elias Kuiter}
\lecture{Product-Line Testing}{testing}

\begin{frame}{Recap: Quality Assurance \deutschertitel{Qualitätssicherung} \mytitlesource{\ludewiglichter}}
	\begin{fancycolumns}[widths={60},animation=none]
		%\only<1|handout:0>{\includegraphics[width=\linewidth,page=1]{quality-assurance}}%
		\only<1|handout:0>{\picDark[width=\linewidth,page=8]{quality-assurance}}%
		\only<2|handout:0>{\picDark[width=\linewidth,page=3]{quality-assurance}}%
		\only<3|handout:0>{\picDark[width=\linewidth,page=4]{quality-assurance}}%
		\only<4|handout:1>{\picDark[width=\linewidth,page=5]{quality-assurance}}%
		\only<5-|handout:0>{\picDark[width=\linewidth,page=7]{quality-assurance}}%
	\nextcolumn
		\begin{note}{Lectures on Quality Assurance}
			how to \emph{avoid} variability bugs\\(esp. feature interactions) \ldots
			\begin{itemize}
				\item<+-> with processes\mysource{\lectureprocess}\\(e.g., domain scoping)
				\item<+-> with guidelines\mysource{\lectureinteractions}
			\end{itemize}
			\visible<3->{how to \emph{find} variability bugs \ldots}
			\begin{itemize}
				\item<+-> \emph{statically}\mysource{\lectureanalyses}
				\item<+-> \emph{dynamically}\mysource{\lecturetesting}\\
				\begin{itemize}
					\item<+-> challenges of product-line testing in Part~\parta
					\item<+-> black-box testing in Part~\partb
					\item<+-> white-box testing in Part~\partc
				\end{itemize}
			\end{itemize}
		\end{note}
	\end{fancycolumns}
\end{frame}

% TODO add xkcd from SWT? \widexkcd{974} % salt 12s

\section{Challenges of Product-Line Testing}
% TODO \section{Product-Line Testing in Practice} practice part still missing
\input{content/11a-practice}
\lessonslearned{
	\item recap on software testing and test-case design
	\item testing all configurations
	\item testing one configuration
	\item sample-based testing
}{
	\item \samplingsurvey: overview on sampling literature
	\item \samplingsurveydatabase: database on sampling algorithms and evaluations
}{
	Recap on feature interactions: What are examples of interactions that cannot be detected statically (cf.\ \lectureanalyses) and could be missed when testing a single configuration only?
}

\section{Combinatorial Interaction Testing}
\begin{frame}{Recap: Black-Box Testing \deutschertitel{Funktionstest}}
	\begin{fancycolumns}[widths={40}]
		\begin{note}{Motivation \mysource{\ludewiglichter}}
			\begin{itemize}
				\setlength\itemsep{.1em}
				\item source code not always available (e.g., outsourced components, obfuscated code)
				%\item specific test cases derived from logical ones using arbitrary values
				%\item specification not incorporated so far (only for expected results)
				%\item invalid inputs not tested
				\item errors are not equally distributed
			\end{itemize}
		\end{note}
		\pause
		\begin{definition}{Black-Box Testing \mysource{\ludewiglichter}}
			\begin{itemize}
				\setlength\itemsep{.1em}
				\item test-case design based on specification
				\item source code and its inner structure is ignored (assumed as a black-box)
			\end{itemize}
		\end{definition}
	\nextcolumn
		\pause
		\begin{note}{Sample Configuration $\neq$ Test Case}
			\begin{itemize}
				\item test case: concrete inputs and expected outputs for a program
				\item sample configuration: selection of features to derive the program
				\item both needed when testing product lines
				\item often confused in the literature
				\item test case derivation
					\begin{itemize}
						\item out of scope here % TODO add pointers to literature
						\item global tests (i.e., identical for all configurations)
						\item product-line implementation technique used to automatically derive configuration-specific tests \mysource{\lectureprocess}
					\end{itemize}
				\item on next slides: idea of black-box testing applied to derive sample configuration
			\end{itemize}
		\end{note}
	\end{fancycolumns}
\end{frame}

\subsection{Pairwise Interaction Testing}
\begin{frame}{\myframetitle{} \deutschertitel{Paarweises Interaktionstesten}}
	\begin{fancycolumns}
		\begin{example}{Configurations with the Interaction Get $\wedge$ Put}
			\footnotesize
			\begin{fancycolumns}[animation=none,widths={55,45}]
				$\{C,G,W\}$\\
				$\{C,P,W\}$\\
				\emph{$\{C,G,P,W\}$}\\
				$\{C,D,W\}$\\
				$\{C,G,D,W\}$\\
				$\{C,P,D,W\}$\\
				\emph{$\{C,G,P,D,W\}$}\\
				$\{C,P,T,W\}$\\
				\emph{$\{C,G,P,T,W\}$}\\
				$\{C,D,T,W\}$\\
				$\{C,G,D,T,W\}$\\
				$\{C,P,D,T,W\}$\\
				\emph{$\{C,G,P,D,T,W\}$}
			\nextcolumn
				$\{C,G,L\}$\\
				$\{C,P,L\}$\\
				\emph{$\{C,G,P,L\}$}\\
				$\{C,D,L\}$\\
				$\{C,G,D,L\}$\\
				$\{C,P,D,L\}$\\
				\emph{$\{C,G,P,D,L\}$}\\
				$\{C,P,T,L\}$\\
				\emph{$\{C,G,P,T,L\}$}\\
				$\{C,D,T,L\}$\\
				$\{C,G,D,T,L\}$\\
				$\{C,P,D,T,L\}$\\
				\emph{$\{C,G,P,D,T,L\}$}
			\end{fancycolumns}
		\end{example}
		\pause
		\begin{definition}{Pairwise Interaction Testing}
			\begin{itemize}
				\setlength\itemsep{.5em}
				\item create a sample $S \subseteq C$, in which every pairwise interaction is covered by at least one configuration %(all valid configurations $C$)
				\item test every configuration in $S$
			\end{itemize}
		\end{definition}
	\nextcolumn
		\pause
		\begin{definition}{Pairwise Combinations}
			\begin{itemize}
				\setlength\itemsep{.5em}
				\item four combinations between $A$ and $B$
				\begin{itemize}
					\item both selected: $A \wedge B$
					\item one selected: $\neg A \wedge B$ and $A \wedge \neg B$
					\item none selected: $\neg A \wedge \neg B$
				\end{itemize}
				\end{itemize}
		\end{definition}
		\pause
		\begin{note}{Discussion}
			\begin{itemize}
				\setlength\itemsep{.4em}
				\item applicable to large product lines
				\item reduced redundant effort compared to\\testing all configurations
				\item full coverage guarantee\\(opposed to random configurations)
				\vspace*{1ex}
				\item still requires good test cases (program inputs)
				\item hard to compute small sample sets
			\end{itemize}
		\end{note}
	\end{fancycolumns}
\end{frame}

\newcommand{\pair}[2]{$#1 \wedge #2$ & $#1 \wedge \neg #2$ & $\neg #1 \wedge #2$ & $\neg #1 \wedge \neg #2$\\}
%\newcommand{\redandgray}[1]{\only<#1-| handout:#1->{\color{black}}\only<#1| handout:#1>{\color{lectureblue}}}
\newcommand{\epairone}[6]{
	\alt<#5->{\textcolor{#3}{$\mathbf{#1} \wedge \mathbf{#2}$}}{$\mathbf{#1} \wedge \mathbf{#2}$} & 
	\alt<#6->{\textcolor{#4}{$\mathbf{#1} \wedge \mathbf{\neg #2}$}}{$\mathbf{#1} \wedge \neg \mathbf{#2}$} & 
}
\newcommand{\epairtwo}[6]{
	\alt<#5->{\textcolor{#3}{$\mathbf{\neg #1} \wedge \mathbf{#2}$}}{$\mathbf{\neg #1} \wedge \mathbf{#2}$} & 
	\alt<#6->{\textcolor{#4}{$\mathbf{\neg #1} \wedge \mathbf{\neg #2}$}}{$\mathbf{\neg #1} \wedge \mathbf{\neg #2}$} \\
}

\begin{frame}[label=PairwiseCoverage]{Pairwise Coverage \deutschertitel{Paarweise Überdeckung}}
	\definecolor{interactionA}{RGB}{178,24,43}
	\definecolor{interactionB}{RGB}{239,138,98}
	\definecolor{interactionC}{RGB}{128,128,128}
	\definecolor{interactionD}{RGB}{0,0,0}
	\definecolor{interactionE}{RGB}{103,169,207}
	\definecolor{interactionF}{RGB}{33,102,172}
	\begin{fancycolumns}[animation=none]
		\centering
		\alt<1>{\featureDiagramConfigurableDatabase}{%
		\alt<2>{\featureDiagramConfigurableDatabaseNoAbstract}
		{\featureDiagramConfigurableDatabaseNoAbstractNoCore}}

		\pause
		\begin{definition}{Interactions to Cover}
			\begin{itemize}
				\setlength\itemsep{.5em}
				\item<2-> exclude abstract features (e.g., $API$, $OS$)
				\item<3-> exclude features contained in every configuration (e.g., $C$)
				\vspace*{1ex}
				\item<4-> exclude invalid combinations (e.g., $W \wedge L$)
			\end{itemize}
			% TODO \todo{add formal definitions based on \lecturemodeling}
		\end{definition}
	\nextcolumn
		\vspace{-10mm}
		\pause\pause
		\begin{example}{Pairwise Interactions}
			\centering\footnotesize\color{lightgray}
			\begin{tabular}{llll}				
				\epairone{G}{P}{interactionC}{interactionB}{7}{6}\epairtwo{G}{P}{interactionA}{interactionF}{5}{10}
				\epairone{G}{D}{interactionB}{interactionC}{6}{7}\epairtwo{G}{D}{interactionA}{interactionE}{5}{9}
				\epairone{G}{T}{interactionC}{interactionB}{7}{6}\epairtwo{G}{T}{interactionA}{interactionE}{5}{9}
				\epairone{G}{W}{interactionD}{interactionB}{8}{6}\epairtwo{G}{W}{interactionA}{interactionF}{5}{10}
				\epairone{G}{L}{interactionB}{interactionD}{6}{8}\epairtwo{G}{L}{interactionF}{interactionA}{10}{5}
				\epairone{P}{D}{interactionA}{interactionC}{5}{7}\epairtwo{P}{D}{interactionB}{interactionD}{6}{8}
				\epairone{P}{T}{interactionA}{interactionE}{5}{9}\epairtwo{P}{T}{interactionF}{interactionB}{10}{6}
				\epairone{P}{W}{interactionA}{interactionC}{5}{7}\epairtwo{P}{W}{interactionD}{interactionB}{8}{6}
				\epairone{P}{L}{interactionC}{interactionA}{7}{5}\epairtwo{P}{L}{interactionB}{interactionD}{6}{8}
				\epairone{D}{T}{interactionA}{interactionB}{5}{6}\epairtwo{D}{T}{interactionC}{interactionD}{7}{8}
				\epairone{D}{W}{interactionA}{interactionB}{5}{6}\epairtwo{D}{W}{interactionD}{interactionC}{8}{7}
				\epairone{D}{L}{interactionB}{interactionA}{6}{5}\epairtwo{D}{L}{interactionC}{interactionD}{7}{8}
				\epairone{T}{W}{interactionA}{interactionC}{5}{7}\epairtwo{T}{W}{interactionD}{interactionB}{8}{6}
				\epairone{T}{L}{interactionC}{interactionA}{7}{5}\epairtwo{T}{L}{interactionB}{interactionD}{6}{8}
				&  \alt<6->{\textcolor{interactionB}{$\mathbf{L} \wedge \mathbf{\neg W}$}}{$\mathbf{L} \wedge \mathbf{\neg W}$} & \alt<5->{\textcolor{interactionA}{$\mathbf{\neg L} \wedge \mathbf{W}$}}{$\mathbf{\neg L} \wedge \mathbf{W}$} & \\
			\end{tabular} 
		\end{example}
		\pause
		\begin{example}{Pairwise Coverage with Six Configurations}
			\footnotesize\color{lightgray}
			\alt<5->{\textcolor{interactionA}{$\mathbf{\{C,P,D,T,W\}}$}}{$\mathbf{\{C,P,D,T,W\}}$}\\
			\alt<6->{\textcolor{interactionB}{$\mathbf{\{C,G,D,L\}}$}}{$\mathbf{\{C,G,D,L\}}$}\\
			\alt<7->{\textcolor{interactionC}{$\mathbf{\{C,G,P,T,L\}}$}}{$\mathbf{\{C,G,P,T,L\}}$}\\
			\alt<8->{\textcolor{interactionD}{$\mathbf{\{C,G,W\}}$}}{$\mathbf{\{C,G,W\}}$}\\
			\alt<9->{\textcolor{interactionE}{$\mathbf{\{C,P,W\}}$}}{$\mathbf{\{C,P,W\}}$}\\
			\alt<10->{\textcolor{interactionF}{$\mathbf{\{C,D,T,L\}}$}}{$\mathbf{\{C,D,T,L\}}$}\\
		\end{example}
	\end{fancycolumns}
\end{frame}

\subsection{T-Wise Interaction Testing}
\begin{frame}{\myframetitle{}}
	\begin{fancycolumns}[widths={60,38}]
		\begin{definition}{T-Wise Interaction Testing}
			\begin{itemize}
				\setlength\itemsep{.5em}
				\item generalization of pairwise interaction testing
				\item t-wise coverage: every t-wise interaction is covered by at least one configuration in the sample
				\item $t=1$: every feature is selected and also deselected
				\item $t=2$: pairwise interaction coverage
				\item $t=3$: every valid combination of three features covered
			\end{itemize}
		\end{definition}
	\nextcolumn
		\begin{example}{{$t=3$ Interactions}}
			for the features $G$, $P$, and $D$:

			\begin{fancycolumns}[animation=none]
				$G \wedge P \wedge D$\\
				$G \wedge P \wedge \neg D$\\
				$G \wedge \neg P \wedge D$\\
				$G \wedge \neg P \wedge \neg D$
			\nextcolumn
				$\neg G \wedge P \wedge D$\\
				$\neg G \wedge P \wedge \neg D$\\
				$\neg G \wedge \neg P \wedge D$\\
				$\neg G \wedge \neg P \wedge \neg D$
			\end{fancycolumns}
		\end{example}
	\end{fancycolumns}
\end{frame}

\subsection{Algorithms for Combinatorial Interaction Testing}
\begin{frame}{\myframetitle{} \deutschertitel{Algorithmen für kombinatorisches Interaktionstesten}}
	\begin{fancycolumns}[widths={63}]
		\begin{definition}{A Greedy Algorithm}
			idea: select configuration that cover most missing interactions in each step
			\begin{enumerate}
				\item randomly choose first configuration
				\item find next optimal configuration
				\item repeat step 2 until all interactions are covered
			\end{enumerate}
		\end{definition}
		\pause
		\begin{note}{Challenges and Optimizations}
			\begin{itemize}
				\item non-deterministic: different sample for each run (cf.\ Step~1)
				\begin{itemize}
					\item starting with all-yes-config? $\Rightarrow$ covers more code
				\end{itemize}
				\item iterating all valid configurations does not scale (cf.\ Step~2)
				\item greedy strategy: optimal configuration in each step does not guarantee optimal sample
			\end{itemize}
		\end{note}
	\nextcolumn
		\pause
		\begin{definition}{ICPL\mysource{\icpl}}
			\begin{itemize}
				\item widespread greedy algorithm
				\item iterates over all interactions
				\begin{itemize}
					\item identifies core and dead features early
					\item identifies invalid and already covered interactions
					\item utilizes parallelization
				\end{itemize}
				\item incrementally increases $t$ up to desired value
				\item performance shown on next slides
			\end{itemize}
		\end{definition}
	\end{fancycolumns}
\end{frame}

\subsection{Efficiency of Combinatorial Interaction Testing}
%\subsection{Combinatorial Interaction Testing with ICPL}
\begin{frame}{\myframetitle{} \mytitlesource{\icpl}}
	\begin{exampletight}{Assumption: All Features are Optional}
		\centering\footnotesize\featureDiagramEightOptionalFeatures
	\end{exampletight}
	\vspace{-2mm} % TODO Benno: not sure why this is needed
	\begin{fancycolumns}
		\pause
		\begin{exampletight}{Number of Configurations in Pairwise Sample}
			\pic[width=\linewidth,page=4]{cit-plots}
		\end{exampletight}
	\nextcolumn
		\pause
		\begin{exampletight}{Number of Configurations in T-Wise Sample}
			\pic[width=\linewidth,page=5]{cit-plots}
		\end{exampletight}
	\end{fancycolumns}
\end{frame}

\begin{frame}{\myframetitle{} \mytitlesource{\icpl}}
	\begin{fancycolumns}[t]
		\begin{exampletight}{Time in Minutes to Compute Sample}
			\pic[width=\linewidth,page=2]{cit-plots}

			\begin{itemize}
				\setlength\itemsep{.5em}
				\item about 9h for Linux
				\item 480 configuration in pairwise sample
			\end{itemize}
		\end{exampletight}
	\nextcolumn
		\begin{exampletight}{Number of Configurations in Sample}
			\pic[width=\linewidth,page=3]{cit-plots}

			\begin{itemize}
				\setlength\itemsep{.5em}
				\item Linux kernel v2.6.28.6 (February 2009)
				\item 6,888 features
				\item 187,193 clauses in conjunctive normal form
			\end{itemize}
		\end{exampletight}
	\end{fancycolumns}
\end{frame}

% TODO distinguish testing efficiency and sampling efficiency

\subsection{Effectiveness of Combinatorial Interaction Testing}
\begin{frame}{\myframetitle{}}
	\begin{fancycolumns}
		\begin{exampletight}{Effectiveness of Interaction Testing \mysource{\href{https://ieeexplore.ieee.org/document/1321063}{Kuhn et al.\ 2004}}}
			\pic[width=\linewidth,page=1]{cit-plots}
		\end{exampletight}
	\nextcolumn
		\begin{note}{Trade-Off}
			large t: high coverage (more effective)
			
			small t: low testing effort (more efficient)
		\end{note}
	\end{fancycolumns}
\end{frame}

\lessonslearned{
	\item recap on black-box testing
	\item combinatorial interaction testing: pairwise testing, t-wise testing
	\item efficiency: number of configurations, time to compute sample
	\item effectiveness: percentage of found defects
}{
	\item \icpl{} -- popular t-wise sampling algorithm ICPL
	\item \yasa{} -- alternative t-wise sampling algorithm YASA
}{
	Why is it hard to find a good trade-off between efficiency and effectiveness?
}

\section{Solution-Space Sampling}
\subsection{Coverage in Single-System Engineering}
\begin{frame}{Recap: Coverage in White-Box Testing{} \deutschertitel{Testüberdeckung für Strukturtests} \mytitlesource{\ludewiglichter}}
	\begin{fancycolumns}
		\begin{definition}{White-Box Testing \deutsch{Strukturtest}}
			\begin{itemize}
%				\setlength\itemsep{.1em}
				\item inner structure of test object is used
				\item idea: coverage of structural elements
				\begin{itemize}
					\item code translated into control flow graph
					\item specific test case (concrete inputs)\\derived from logical test case (conditions)\\derived from path in control flow graph
				\end{itemize}	
			\end{itemize}
		\end{definition}
	\nextcolumn
		\begin{definition}{Coverage Criteria \deutsch{Überdeckungskriterien}}
			\begin{itemize}
%				\setlength\itemsep{.1em}
				\item[1.] \emph{statement coverage} \deutsch{Anweisungsüberdeck.}:\\all statements are executed for at least one test case
				\item<3->[2.] \emph{branching coverage} \deutsch{Zweigüberdeckung}: statement coverage and all branches of branching statement are executed %TODO not so easy to define as percentage
				\item<4->[3.] \emph{term coverage} \deutsch{Termüberdeckung}:\\branching coverage and all terms used in a branching statement ($n$) are combined exhaustively ($2^n$)\hfill(simplified)
				% TODO discuss path coverage?
			\end{itemize}
		\end{definition}
	\end{fancycolumns}
\end{frame}

\subsection{Coverage of Ifdef Blocks}
\begin{frame}[b]
	\mywhite{Can You Spot Problems in the Elevator Product Line?\mysource{\samplingsurvey}}{
		\centering\pic[width=.8\linewidth,page=3,trim=40 530 70 100,clip]{2018/2018-SPLC-Varshosaz}
	}
	\pause
	\begin{itemize}
		\item<+-> Line~29: compiler error when $FIFO$: field \texttt{dreq} undefined
		\item<+-> Line~29: compiler error when $FIFO \pand \pnot DirectedCall$: method \texttt{callButtonsNextState} undefined
		\item<+-> Line~8: runtime error when $DirectedCall \pand \pnot FIFO$: null pointer exception
		\item<+-> both problems detectable with pairwise coverage, but presence conditions are more complicated in practice
		\item<+-> also: pairwise coverage often too much effort for large configuration spaces / continuous integration
	\end{itemize}
	\onslide<+->{\begin{definition}{\myframetitle{} \deutschertitel{Überdeckung von Ifdef-Blöcken} \mysource{\tartlerconfigurationcoverage}}
	\begin{itemize}
		\item every block selected for at least one configuration in the sample (cf.\ statement coverage)
	\end{itemize}
	\end{definition}}
\end{frame}

\subsection{Presence-Condition Coverage}
\begin{frame}{\myframetitle{} \deutschertitel{Presence-Condition-Überdeckung}}
	\begin{fancycolumns}[widths={48}]
		\begin{definition}{Presence-Condition Coverage\mysource{\krieterpresenceconditioncoverage}}
			\begin{itemize}
				\item application of t-wise interaction testing to presence conditions
				\item recap presence condition: formula specifying exactly those configurations under which a block is present
				\item t-wise presence condition coverage: every t-wise interaction of presence conditions is covered by at least one configuration in the sample
				\item $t=1$: every block is selected and also deselected (i.e., more than Tartler's coverage of ifdef blocks)
				\item $t=2$: every combination of two blocks covered
				\item $t=3$: every combination of three blocks covered
			\end{itemize}
		\end{definition}
	\nextcolumn
		\pause
		\begin{example}{{T=3 Presence-Condition Interactions}}
			for the blocks $a$, $b$, and $c$ with presence conditions $A$, $B$, and $C$:

			\begin{fancycolumns}[animation=none]
				$A \wedge B \wedge C$\\
				$A \wedge B \wedge \neg C$\\
				$A \wedge \neg B \wedge C$\\
				$A \wedge \neg B \wedge \neg C$
			\nextcolumn
				$\neg A \wedge B \wedge C$\\
				$\neg A \wedge B \wedge \neg C$\\
				$\neg A \wedge \neg B \wedge C$\\
				$\neg A \wedge \neg B \wedge \neg C$
			\end{fancycolumns}
		\end{example}
		\pause
		\begin{note}{Presence-Condition Coverage\mysource{\krieterpresenceconditioncoverage}}
			\begin{itemize}
				\item coverage of solution space (not problem space)
				\item aka.\ solution-space sampling
				\item for same t: often fewer configurations and similar effectiveness than feature interaction coverage
				\item also feasible by translating presence conditions into feature model \mysource{\hentzesolutionspacesampling}
			\end{itemize}
		\end{note}
	\end{fancycolumns}
\end{frame}

\subsection{Overview on Coverage Criteria}
\begin{frame}{\myframetitle{} \deutschertitel{Überblick über Überdeckungskriterien}}
	\begin{definition}{Techniques \& Coverage Criteria \mysource{\samplingsurvey}}
		\pic[width=\linewidth,page=4,trim=50 100 310 610,clip]{2018/2018-SPLC-Varshosaz}
	\end{definition}
\end{frame}

\subsection{Input for Sampling Algorithms}
\begin{frame}{\myframetitle{}}
	\begin{fancycolumns}
		\begin{definition}{Input Data \mysource{\samplingsurvey}}
			\pic[width=\linewidth,page=3,trim=350 430 90 310,clip]{2018/2018-SPLC-Varshosaz}
		\end{definition}
		\pause
		\begin{example}{Further Domain Knowledge \mysource{\samplingsurvey}}
			\begin{itemize}
				\item in addition to feature model
				\item e.g., configurations chosen by experts
				\item e.g., specialized feature model for sampling
			\end{itemize}
		\end{example}
	\nextcolumn
		\pause
		\vspace{-5mm}
		\begin{example}{Part 2: Combinatorial Interaction Testing}
			\begin{itemize}
				\item (Problem-Space Sampling)
				\item \emph{feature model} used to consider only valid configurations
			\end{itemize}
		\end{example}
		\pause
		\begin{example}{Part 3: Solution-Space Sampling}
			\begin{itemize}
				\item mapping from features to \emph{implementation artifacts}
				\item \emph{feature model} used to consider only valid configurations
			\end{itemize}
		\end{example}
		\pause
		\begin{example}{Combinatorial Reduction of Tests \mysource{\reducingconfigurations}}
			\begin{itemize}
				\item which configurations matter for each test?
				\item analyze \emph{unit tests} and \emph{impl. artifacts}
				\item \emph{feature model} used to consider only valid configurations
			\end{itemize}
		\end{example}
	\end{fancycolumns}
\end{frame}

\lessonslearned{
	\item recap on white-box testing and coverage criteria
	\item coverage of ifdef blocks
	\item t-wise presence condition coverage
	\item overview on techniques, coverage criteria, input data for sampling
}{
	\item \tartlerconfigurationcoverage: covering every ifdef block (but not their absence)
	\item \krieterpresenceconditioncoverage: solution-space sampling as discussed in this part
	\item \hentzesolutionspacesampling: translation of presence conditions into feature model + reuse of problem-space sampling
}{
	Does the order of configurations matter during testing?\mysource{\alhajjajiprioritization}
}

\faq{
	\item What is the goal of quality assurance for software product lines?
	\item How can product lines be tested?
	\item Why is testing product lines challenging?
	\item What are (dis-)advantages of testing all configurations?
	\item What are (dis-)advantages of testing only one configuration?
	\item What is sample-based testing?
	\item What is a sample?
	\item How can a sample be computed?
}{
	\item How is black-box testing used for testing product lines?
	\item What is the difference between a test configuration and a test case?
	\item What are (dis-)advantages of combinatorial interaction testing?
	\item What is pairwise interaction testing?
	\item What is t-wise interaction testing?
	\item When does a sample achieve 100\% pairwise coverage?
	\item How can a t-wise sample be computed?
}{
	\item How can white-box testing be used for testing product lines?
	\item What are potential problems with t-wise interaction testing?
	\item What is presence condition coverage?
	\item What are different techniques for t-wise sampling?
	\item Which additional inputs can be used for sampling algorithms?
	\item How efficient and how effective are sampling algorithms?
}

\mode<beamer>{
	\addtocounter{framenumber}{-1}
	\begin{frame}{\inserttitle}
		\lectureseriesoverview[\insertlecturenumber]
	\end{frame}

	%\addtocounter{framenumber}{-1}
	%\againtitle % TODO does not work as we have redefined maketitle
}


% TODO L12 EVOLUTION AND MAINTENANCE

\ifuniversity{recording}{\date{June 28, 2023}\setpicture[275]{may23-west1}\setcopyright{}}
\ifuniversity{ulm}{\date{July 18, 2023}\setpicture[275]{may23-west1}}
\ifuniversity{paderborn}{\date{July 03, 2024}\setpicture[125]{pressimage3}}
\ifuniversity{braunschweig}{\date{January 15, 2025}}

\author{Thomas Thüm, Elias Kuiter, Timo Kehrer}
\lecture{Evolution and Maintenance}{evonance}

\section{Product-Line Evolution}
\input{content/12a-evolution}
\lessonslearned{
	\item changes are inevitable and occur frequently
	\item product lines typically grow over time
	\item different kinds of changes to feature models (e.g., refactorings)
	\item co-evolution of feature model, feature mapping, artifacts
}{
	\item \reasoningfme\\ --- SAT-based classification of feature-model changes
	\item \fmrefactoring\\ --- patterns for feature-model refactorings
	\item \splevolution\\ --- product-line evolution
	% TODO literature presenting laws to translate tree constraints into cross-tree constraints and back https://www.jucs.org/jucs_14_21/algebraic_laws_for_feature/jucs_14_21_3573_3591_gheyi.pdf
}{
	After which changes do we need to analyze and test the product line again?

	Do we always need to analyze and test the complete product line again?
}

\section{Product-Line Maintenance}
\input{content/12b-maintenance}
\lessonslearned{
	\item what is maintenance?
	\item what are examples for product-line maintenance?
	\item what is reengineering?
	\item what are examples of product-line reengineering?
}{
	\item \ludewiglichter\\ --- maintenance and reengineering
}{
	How do \emph{product-line adoption strategies}

	(proactive, reactive, extractive)

	correlate with \emph{reengineering tasks}

	(reengineering, refactoring, forward/reverse engineering)?
}

\section{Course Summary}
% TODO no outlook yet: \section{Course Summary and Outlook}
\input{content/12c-summary}
\lessonslearned{
	\item how to implement features and variability?
	\item how to model valid combinations and reason about those?
	\item how to do quality assurance?
	\item (how to evolve and maintain a product line?)
}{
	\item[] see earlier parts
}{
	Questions? Feedback?
}

\faq{
	\item Why do we need to change product lines?
	\item How does the Linux kernel evolve over time?
	\item How can changes to feature models be classified?
	\item What are advantages of classifying changes to feature models?
	\item Give an example for a refactoring/generalization/specialization!
	\item What is referred to as co-evolution of product lines?
	\item How do feature model, build scripts, and source code co-evolve?
}{
	\item What is the difference between evolution and maintenance?
	\item Which kinds of maintenance and reengineering are there?
	\item What are examples for product-line maintenance?
	\item What are examples for adaptive, corrective, perfective, preventive maintenance?
	\item What are examples for reverse engineering, forward engineering, reengineering?
}{
	\item What is a software product line?
	\item When to use which implementation technique for variability?
	\item How to perform quality assurance for product lines?
}

\mode<beamer>{
	\addtocounter{framenumber}{-1}
	\begin{frame}{\inserttitle}
		\lectureseriesoverview[\insertlecturenumber]
	\end{frame}

	%\addtocounter{framenumber}{-1}
	%\againtitle % TODO does not work as we have redefined maketitle
}

