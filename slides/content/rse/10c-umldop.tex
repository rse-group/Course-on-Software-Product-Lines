\begin{frame}{Motivation}
	\begin{fancycolumns}
	\begin{note}{}
		\begin{itemize}
			\item \textbf{Delta-oriented programming} is a more flexible version of \textbf{feature-oriented programming}
			\newline DOP is an object-oriented paradigm compatible with mainstream programming languages and development processes
			\item \textbf{Traceability} between problem space and solution space 
			\newline Delta modules implement a certain feature, and what features are implemented by a certain 
				delta module. Software product can be generated automatically
				by tracing the corresponding delta modules and composing them
			\item Need for a \textbf{visual design language} for delta-oriented SPL 
			\newline Deltas and features can be composed in various ways to obtain all products of the SPL.
			The visual design to aid this process is highly useful but has been missing so far from the delta-oriented SPL development process.
		\end{itemize}
	\end{note}
	\pause
	\begin{note}{Problem}
		 UML, as a standard approach, does not include modeling elements for specifying the structural variability of a software design.
	\end{note}

	\end{fancycolumns}
\end{frame}




\begin{frame}{UML Profile}
	\begin{fancycolumns}[widths={30},animation=none]
	
	\end{fancycolumns}

\end{frame}



\begin{frame}{UML Profile for DOP}
	\begin{fancycolumns}[widths={30},animation=none]
		
	\end{fancycolumns}
\end{frame}




\begin{frame}{Examples}
	\begin{fancycolumns}[widths={30},animation=none]
		
	\end{fancycolumns}
\end{frame}

\mode<beamer>{
	\addtocounter{framenumber}{-1}
	\begin{frame}{\inserttitle}
		\lectureseriesoverview[\insertlecturenumber]
	\end{frame}

	%\addtocounter{framenumber}{-1}
	%\againtitle % TODO does not work as we have redefined maketitle
}

