% TODO add this paper? https://www.sciencedirect.com/science/article/pii/S0950584916300921
% TODO add examples for undisciplined ifdefs? https://dl.acm.org/doi/abs/10.1145/1960275.1960299

\subsection{Granularity of Variability}
\begin{frame}{\myframetitle}
	\begin{fancycolumns}[animation=none]
		\begin{definition}{Granularity of Variability}
			\begin{itemize}
				\item Depending on the implementation technique, variability can be introduced at different levels of granularity.
				\item A level of granularity refers to
				\begin{itemize}
					\item the hierarchical organization of implementation artifacts (e.g., through the file system),
					\item the hierarchical structure of an implementation artifact (e.g., given by its syntax)
				\end{itemize}
			\end{itemize}
		\end{definition}
		\begin{example}{Granularity Levels in Java}
			modules $>$ libraries $>$ packages $>$ classes $>$ members $>$ statements $>$ parameters
		\end{example}
		\pause
	\nextcolumn
		\begin{note}{What we have seen so far?}
			\begin{itemize}
				\item Coarse-grained: Clone-and-own with version control (entire variants)
				\item Medium-grained: Clone-and-own with build systems (file level)
				\item Medium-grained: Features with build systems (file level)
				\item Medium-grained: Design patterns for variability (class or member level)
				\item Fine-grained: Runtime parameters (statement level) 
			\end{itemize}
		\end{note}
		\pause
		\begin{note}{What is missing?}
			Yet no approach supporting fine-grained compile-time variability!
		\end{note}		
	\end{fancycolumns}
\end{frame}
% Christian's paper?
% empirical studies?
% essence: file-level variability is not engough

\subsection{What is a Preprocessor?}
\begin{frame}{\myframetitle\ \mytitlesource{\fospl\mypages{110--111}}}
	\begin{fancycolumns}
		\begin{definition}{Preprocessor}
			\begin{itemize}
				\item tool manipulating source code before compilation (i.e., at compile time)
				\item preprocessors are used:
					\begin{itemize}
						\item to inline files\hfill(e.g., header files)
						\item to define and expand macros\\\hfill(cf.\ metaprogramming)
						\item for \textbf{conditional compilation}\\\hfill(e.g., remove debug code for release)
					\end{itemize}
			\end{itemize}
		\end{definition}
	\nextcolumn
		\begin{note}{Preprocessor}
			\begin{itemize}
				\item the C Preprocessor (CPP) is used in almost every C/C++ project
				\item preprocessors are typically oblivious to the target language as they operate on text files (e.g., the C Preprocessor can also used for Fortran or Java)
				\item conditional compilation is a very common technique to implement product lines
			\end{itemize}
		\end{note}
	\end{fancycolumns}
\end{frame}

% in-place vs outa-place preprocessors
% how to select features? parameters passed to the preprocessor, define in source code

\subsection{CPP -- The C Preprocessor}
\begin{frame}[label=SPLwithPreprocessors]{\myframetitle}
	\setlength\leftmargini{4mm}%
	\begin{fancycolumns}[T,columns=3,animation=none]
		\begin{definition}{CPP Directives\mysource{\href{https://en.cppreference.com/w/cpp/preprocessor}{cppreference.com}}}
			file inclusion
			\begin{itemize}
				\item \texttt{\#include}
			\end{itemize}
			\only<2->{text replacement
			\begin{itemize}
				\item \texttt{\#define}
				\item \texttt{\#undef}
			\end{itemize}}
			\only<3->{conditional compilation
			\begin{itemize}
				\item \texttt{\#if}, \texttt{\#endif}
				\only<6->{\item \texttt{\#else}, \texttt{\#elif}
				\item<7-> \texttt{\#ifdef}, \texttt{\#ifndef}
				\item<8-> new: \texttt{\#elifdef}, \texttt{\#elifndef}}
			\end{itemize}}
		\end{definition}
	\nextcolumn
		\begin{exampletight}{Example Input}
			\picDark[scale=.3]{preprocessor-c}
		\end{exampletight}
	\nextcolumn
		\uncover<4->{
			\begin{exampletight}{Example Output (Simplified)}
				\picDark[scale=.15]{preprocessor-c-output}
			\end{exampletight}
		}
		\uncover<5->{\begin{note}{Why simplified?}
			\begin{itemize}
				\item preprocessed file can get very long due to to included header files
				\item preprocessors typically do not remove line breaks to not influence line numbers reported by compilers
			\end{itemize}
		\end{note}}
	\end{fancycolumns}
\end{frame}
% TODO example contains "Beautiful" but should be "beautiful"
% TODO illustrate parameters and the call of the preprocessor

% TODO explain the most important commands: #if #defined or and not ... #elif

% TODO #error
% TODO parameters, #define, even combinations

% TODO single characters
% TODO discipliced, undisciplined

\subsection{Preprocessors for Java}

\begin{frame}{Munge -- A Simple Preprocessor for Java \mytitlesource{\featureide}}
	\begin{fancycolumns}[T,widths={75}]
		\begin{exampletight}{Example Input and Output}
			\centering\picDark[width=\linewidth]{preprocessor-munge}
		\end{exampletight}
	\nextcolumn
		\begin{note}{Munge}
			\begin{itemize}
				\item preprocessor for Java
				\item written in Java
				\item about 300 LOC
			\end{itemize}
		\end{note}
	\end{fancycolumns}

	\begin{exampletight}{Call of Munge on Command Line}
		\centering\picDark[scale=.3]{preprocessor-munge-call}
	\end{exampletight}
\end{frame}

\begin{frame}{Antenna -- An In-Place Preprocessor for Java \mytitlesource{\featureide}}
	\begin{fancycolumns}[T,widths={75}]
		\begin{exampletight}{Example Input and Output}
			\centering\picDark[width=\linewidth]{preprocessor-antenna}
		\end{exampletight}
	\nextcolumn
		\begin{note}{Antenna}
			\begin{itemize}
				\item preprocessor for Java
				\item written in Java
				\item has been used for Java ME (micro edition) projects
			\end{itemize}
		\end{note}
	\end{fancycolumns}

	\begin{exampletight}{Call of Antenna on Command Line}
		\centering\picDark[scale=.3]{preprocessor-antenna-call}
	\end{exampletight}
\end{frame}

\begin{frame}{In-Place and Out-of-Place Preprocessors \mytitlesource{\featureide}}
	\begin{fancycolumns}[T]
		\begin{note}{In-Place Preprocessor}
			\begin{itemize}
				\item input file manipulated
				\item lines commented out where necessary
				\item often: better support in IDEs
			\end{itemize}
		\end{note}
		\begin{exampletight}{Antenna, \ldots}
			\picDark[height=30mm]{preprocessor-antenna-idea}
		\end{exampletight}
	\nextcolumn
		\begin{note}{Out-of-Place Preprocessor}
			\begin{itemize}
				\item separate output file generated
				\item lines deleted where necessary
				\item often: worse support in IDEs
			\end{itemize}
		\end{note}
		\begin{exampletight}{CPP, Munge, \ldots}
			\picDark[height=30mm]{preprocessor-munge-idea}
		\end{exampletight}
	\end{fancycolumns}
\end{frame}

\subsection{Preprocessors in FeatureIDE}
\begin{frame}{\myframetitle\ \mytitlesource{\featureide}}
	\begin{fancycolumns}[animation=none,widths={75}]
		\only<1|handout:0>{\pic[width=\linewidth]{featureide-antenna-featuremodel}}%
		\only<2|handout:1>{\pic[width=\linewidth]{featureide-antenna-configuration}}%
		\only<3|handout:0>{\pic[width=\linewidth]{featureide-antenna-warning}}%
		\only<4|handout:0>{\pic[width=\linewidth]{featureide-antenna-contentassist}}%
	\nextcolumn
		\begin{note}{\href{https://www.youtube.com/watch?v=jVe7f32mLCQ}{Demo Video}}
			\begin{itemize}
				\item preprocessing with Antenna on command line
				\item feature modeling
				\item warnings for unreferenced features
				\item content assist proposing feature names
				\item configuration and automated regeneration
				\item (first 2 min relevant here)
			\end{itemize}
		\end{note}
	\end{fancycolumns}
\end{frame}

\subsection[Discussion]{Discussion of Preprocessors}

\begin{frame}{\myframetitle}
	\begin{fancycolumns}[T,animation=none,widths={26}]
		\begin{note}{Powerful Preprocessors}
			we can annotate
			\setlength\leftmargini{3mm}
			\begin{itemize}
				\item complete files (i.e., Java classes)
				\item members (e.g., fields, methods)
				\item (parts of) statements
				\item parameters
				\item \ldots
				\item single characters
			\end{itemize}
			\uncover<2->{and automatically generate variants}
		\end{note}
		\uncover<3->{\begin{example}{}
			everyone happy?
		\end{example}}
	\nextcolumn
		\only<1|handout:0>{\pic[width=\linewidth,page=1,trim=20 20 20 40,clip]{preprocessor-wilderness}}%
		\only<2->{\pic[width=\linewidth,page=2,trim=20 20 20 40,clip]{preprocessor-wilderness}}
	\end{fancycolumns}
\end{frame}

\begin{frame}{\myframetitle}
	\begin{fancycolumns}[T,animation=none,widths={73}]
		\pic[width=\linewidth,page=3,trim=20 20 20 40,clip]{preprocessor-wilderness}
	\nextcolumn
		\begin{example}{Syntax Error}
			missing ) and semicolon!
		\end{example}

		\pause
		\begin{note}{Powerful Preprocessors}
			\setlength\leftmargini{3mm}
			can produce syntactically ill-formed programs:
			\begin{itemize}
				\item errors may appear only for certain configurations,
				\item are hard to find, and
				\item are more likely than for other techniques
			\end{itemize}

			\mysource{more in \lecturefeatures\partc}
		\end{note}
	\end{fancycolumns}
\end{frame}

\begin{frame}{\myframetitle}
	\begin{fancycolumns}[T,animation=none,widths={73}]
		\pic[width=\linewidth,page=4,trim=20 20 20 40,clip]{preprocessor-wilderness}
	\nextcolumn
		\begin{example}{Type Error}
			\texttt{weight} undefined!
		\end{example}

		\pause
		\begin{note}{Powerful Preprocessors}
			\setlength\leftmargini{3mm}
			can produce ill-typed programs:
			\begin{itemize}
				\item errors may appear only for certain configurations,
				\item are hard to find, and
				\item are more likely than for other techniques
			\end{itemize}
			\mysource{more in \lectureanalyses}
		\end{note}
	\end{fancycolumns}
\end{frame}

\begin{frame}{\myframetitle}
	\begin{fancycolumns}[T,animation=none,widths={73}]
		\pic[width=\linewidth,page=5,trim=20 20 20 40,clip]{preprocessor-wilderness}
	\nextcolumn
		\begin{example}{Runtime Error}
			assertion failed!
		\end{example}

		\pause
		\begin{note}{Powerful Preprocessors}
			\setlength\leftmargini{3mm}
			can produce programs with\,unwanted\,behavior:
			\begin{itemize}
				\item errors may appear only for certain configurations,
				\item are hard to find, and
				\item are more likely than for other techniques
			\end{itemize}
			\mysource{more in \lecturetesting}
		\end{note}
	\end{fancycolumns}
\end{frame}

%\begin{frame}{\myframetitle}
%	\leftorright{
%		\myexampletight{A Slightly More Complex Example}{\pic[width=\linewidth]{preprocessor-antenna-elevator}}
%	}{
%		\todots
%	}
%\end{frame}

% pros: fine granular, language-independent
% cons: IDE support, easy to create mistakes

\begin{frame}[fragile]{Problem: Source-Code ``Obfuscation''}
	\begin{fancycolumns}[reverse]
		\begin{note}{Observations on Readability}
			\begin{itemize}
				\item Mixing \emph{two languages} (C and \#ifdefs, or Java and Munge, ...)
				\item Control flow difficult to understand
				\item Long annotations hard to find
				\item Extra line breaks destroy layout
			\end{itemize}
		\end{note}
		\begin{definition}{Problem: Undisciplined Annotations}
			\begin{itemize}
				\item the preprocessor language (e.g., \#ifdefs) does not care about the preprocessed language (e.g., C)
				\item allows for ``undisciplined'' preprocessor usage (precise definition later)
				\item considerably worsens readability
			\end{itemize}
		\end{definition}
	\nextcolumn
		\begin{cpptight}[basicstyle=\small]{Can you read this source code?\mysource{\href{https://dl.acm.org/doi/abs/10.1145/1960275.1960299}{{Liebig~et~al.~2011; xterm}}}}
#if ~defined(__GLIBC__)~
	// additional lines of code
#elif ~defined(__MVS__)~
	result = pty_search(pty);
#else
#ifdef ~USE_ISPTS_FLAG~
	if (result) {
#endif
		result = ((*pty=open("/dev/ptmx", O_RDWR))<0);
#endif
#if ~defined(SVR4) || defined(__SCO__) || \~
		~defined(USE_ISPTS_FLAG)~
		if (!result)
			strcpy(ttydev, ptsname(*pty));
#ifdef ~USE_ISPTS_FLAG~
		IsPts = !result;
	}
#endif
#endif
		\end{cpptight}
	\end{fancycolumns}
\end{frame}

\newcommand{\MajorChallengesOfPreprocessors}{
	\item \emph{scattering} and \emph{tangling}\\
	$\Rightarrow$ separation of concerns?
	\item mixes multiple languages in the same development artifact
	\item may \emph{obfuscate} source code and severely impact its readability
	\item hard to analyze and process for existing IDEs
	\item often used in an ad-hoc or \emph{undisciplined} fashion
	\item prone to subtle syntax, type, or runtime errors which are hard to detect \mysource{\lectureinteractions--\lecturetesting}
}
\begin{frame}<1-2>[label=DiscussionOfPreprocessors]
	\frametitle<1-2>{Discussion of Preprocessors}
	\frametitle<3>{\myframetitle}
	\begin{fancycolumns}
		\begin{note}{Advantages}
			\begin{itemize}
				\item well-known and mature tools, readily available
				\item easy to use\\
				$\Rightarrow$ just annotate and remove
				\item supports \emph{compile-time variability}
				\item flexible, arbitrary levels of \emph{granularity}
				\item can handle code and non-code artifacts (\emph{uniformity})
				\item little \emph{preplanning} required\\
				$\Rightarrow$ variability can be added to an existing project
			\end{itemize}
		\end{note}
	\nextcolumn
		\begin{note}{Challenges}
			\begin{itemize}
				\MajorChallengesOfPreprocessors
			\end{itemize}
		\end{note}
	\end{fancycolumns}
\end{frame}

\subsection{Preprocessor-Based Product Lines in the Wild}
\begin{frame}{\myframetitle\ \mytitlesource{\fortyproductlines}}
	\begin{fancycolumns}
		\begin{exampletight}{Number of Features}
			\centering\pic[width=.8\linewidth,page=1]{fortyproductlines}\\Lines of Code
		\end{exampletight}
	\nextcolumn
		\begin{exampletight}{Percentage of Variable Code}
			\centering\pic[width=.8\linewidth,page=2]{fortyproductlines}\\Lines of Code
		\end{exampletight}
	\end{fancycolumns}
\end{frame}

\begin{frame}{\myframetitle\ \mytitlesource{\fortyproductlines}}
	\begin{fancycolumns}
		\begin{exampletight}{Lines of Variable Code}
			\centering\pic[width=.8\linewidth,page=3]{fortyproductlines}\\Number of Features
		\end{exampletight}
	\nextcolumn
		\begin{exampletight}{Average Nesting Depth}
			\centering\pic[width=.8\linewidth,page=6]{fortyproductlines}\\Number of Features
		\end{exampletight}
	\end{fancycolumns}
\end{frame}

\begin{frame}{\myframetitle\ \mytitlesource{\fortyproductlines}}
	\begin{fancycolumns}
		\begin{exampletight}{Average Number of Feature References}
			\centering\pic[width=.8\linewidth,page=4]{fortyproductlines}\\Number of Features
		\end{exampletight}
	\nextcolumn
		\begin{exampletight}{Average Number of Features per Annotation}
			\centering\pic[width=.8\linewidth,page=5]{fortyproductlines}\\Number of Features
		\end{exampletight}
	\end{fancycolumns}
\end{frame}
% TODO create own plots for the data?
% TODO add pictures from Rodrigues et al. @ INFSOF’16 (Assessing fine-grained feature dependencies): methods with directives vs product lines

